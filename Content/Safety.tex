\chapter{Safety}

\graphicspath{{./Figures/Modeling}}

\begin{itemize}
	\item hardware.
	\begin{itemize}
		\item motors 
			\begin{itemize}
				\item Hip motor is enclosed in the body. 
				\item the knee motors are enclosed in cover.
				\item the wheels motors have a 
			\end{itemize}
		\item wiring 
		\begin{itemize}
			\item where the wires are neatly fastened in a predetermined route so that it wouldn't be caught in the robot movement which would cause damage to the robot and also to protect them from damage.  
			\item the correct type were used to avoid overheating upon the draw of current from the divers.
			\item Rs-485 connection between motors where used to avoid additional cables.
		\end{itemize}
	
		\item proximity sensor
		\begin{itemize}
			\item  placed in the front and back side of the robot would help to detect crashing in trivial positions such as running into a wall and that could be easily prevented by proximity sensor monitoring the distance between the robot the and the obsticals in it direction. 
		\end{itemize}
			
		\item body safety additional parts in case of impact to protect the internal components. 
	\end{itemize}
	\item software.
	
\end{itemize}