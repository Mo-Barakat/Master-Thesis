\chapter{Mechanical Assembly}

\graphicspath{{./Figures/Modeling}}

\begin{itemize}
	\item Intro Overview of the chapter.
	\item Importance of mechanical assembly in the overall project.
\end{itemize}


\section{Components Overview}
\begin{itemize}
\item Detailed description of all mechanical components used in the robot.
\item Source or method of fabrication for each component (e.g., machined parts, 3D printed elements).
\end{itemize}
\section{Assembly Process}
\begin{itemize}
\item Step-by-step explanation of the assembly process.
\item Tools and techniques used in the assembly.
\item Assembly sequence and rationale behind it.
\end{itemize}
\section{Integration of Mechanical and Electronic Systems}
\begin{itemize}
\item Discussion on how mechanical components interface with electronic systems.
\item Challenges faced in integration and strategies employed to overcome them.
\end{itemize}
\section{Troubleshooting and Problem Solving}
\begin{itemize}
\item Discussion of any unexpected challenges or issues faced during assembly.
\item How these issues were diagnosed and resolved.
\end{itemize}
\section{Safety Considerations}
\begin{itemize}
	\item Safety measures taken during the assembly process.
	\item Design considerations for ensuring the operational safety of the robot.
\end{itemize}
\begin{itemize}
	\item \textbf{Discrete Numerical Simulation:} Elaboration on the process of discrete numerical simulation, including the discrete double integration method to arrive at the state vector.
\end{itemize}

\begin{notebox}
	the bolts used the problem to shorten them so that it wouldn't touch the motor 
\end{notebox}

