\chapter{Conclusion and Future Work}

\graphicspath{{./Figures/Modeling}}

\section{Conclusion}
%This section should summarize the main findings of the research and demonstrate how they address the research questions or gaps in knowledge identified in the introduction.
%Summary of Key Findings
In this research, we developed a multi-legged robotic system that can perform complex movements and interact with its environment.
A new design was created from scratch to meet the requirements of the project.
Throughout the design process, the mechanical, electrical, and software requirements of the robot were taken into account.
Modeling of the new design was done to determine the robot's dynamics.
The modeling includes the new robot's kinematics and dynamics integrating the changing location of the center of mass of the robot as well as the changing  moments of inertia of the robot.
the model was used to simulate the robot's motion and control.
the simulation was used to test different control strategies and simulate the dynamic behavior of the robot.

The new design serves as a platform for future research and development.
The new design is modular and can be easily modified to accommodate different requirements.
The new design is also easily repairable.
The simulation results prove the validity of the model and the effectiveness of the control strategies.
The simulation results also show that the robot can change its configuration and maintain its balance.

Different challenges were faced during the development of the robot.
The main challenges were the mechanical design and the control of the new model in different configurations.
There are still some challenges and limitations that need to be addressed in future work.

\section{Future Work}
%This section should outline any future work that could be done to extend the research.
In future work, Refinement of the design for efficiency, stability.
The next generation design can be more compact and lighter.
Modeling as well can still be improved to include more details and more accurate parameters.
Additionally, implementation of more advanced control strategies can be done.
Different advanced capabilities can be integrated into the robot such as autonomous navigation, machine-learning-based control systems, or enhanced interaction with the environment.
The new robot design can take advantage of the two independent legs to perform more complex movements such as bending one knee more than the other in fast and tight turns.
Collaborative and swarm robotics can be explored, where multiple robots can work together to achieve a common goal.