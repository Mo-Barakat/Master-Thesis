\appendix
% Im Anhang kleine römische Zahlen, beginnend ab i
\pagenumbering{roman}

% Hier den Anhang einfügen, z. B.
% \include{Anhang/TollesKapitel} ...





\chapter{Related Works}\label{ch:app_prior_works}
\graphicspath{{./Bilder/prior_works}} 
\begin{figure}[h!]
        \centering
        \includegraphics[width=1\textwidth]
        {schoellig_et_al_connected_papers_a_dyn_sys_pers.png}
        \caption[Connected Papers for "A dynamical systems perspective"]{Connected Papers overview based on the work "A dynamical systems perspective" showing how tightly interconnected papers from Schoellig et al. are on this topic. The authors Schoellig, Pereida, Zhou, Rimalwala, Helwa and Sorocky are seemingly one research group, making up the majority of the more recent papers in that field.}\label{fig:connected_papers}
\end{figure}



\chapter{Method}
% ____________________________________________________________________
% ======================== Spring Damper System =====================
\section{Input Types}\label{sec:app_meth1}
\graphicspath{{./Bilder/transfer_function_estimation}} 
% ====================================================================


% ===============================================================
% =================  INPUT TYPE: STEP Long ===============
\begin{example}[: Step input (long)]\label{ex:step_input_long}
	Consider the same example systems and method as presented in \exref{ex:single_sine_input_01hz}.
    Both systems are excited using a step signal. The final step value is held for 
    \begin{equation}
        t_{\mathrm{step\,hold}} = 100\,s.
    \end{equation} 
     The estimated input transfer is
     \begin{equation}
        \hat{T}^{(1 \rightarrow 2)}(z) = \frac{0.347\,z - 0.345}{z - 0.999} 
     \end{equation}
     \begin{notebox}
     Note that the longer hold duration results in a smaller error gain for slow frequencies but increases the maximum error gain $\|H^{(1 \rightarrow 2)}\|_{\infty}$.
     \end{notebox}

\begin{minipage}{1\textwidth}
\begingroup
	\centering
	\vspace{1em}
	\captionsetup{format=plain, type=figure, labelfont={color=example_font_color, bf}, font={color=example_font_color}}
    \begin{subfigure}[t]{0.495\textwidth}
        \centering
        \includegraphics[width=\textwidth]{/input_types/step 100 time.pdf}
        \caption{System response to a step training input with and without the estimated input-transfer}
        \label{subfig:tf_est_input_step2_time}
    \end{subfigure}
    \hfill    
    \begin{subfigure}[t]{0.495\textwidth}
        \centering
        \includegraphics[width=\textwidth]{/input_types/step 100 bode.pdf}
        \caption{Bode diagram of the error dynamics with and without the estimated input-transfer}
        \label{subfig:tf_est_input_step2_bode}
    \end{subfigure}
    \caption[Training Inputs for Transfer Map Estimation (Step II)]{Estimated input-transfer from a unit step training input which was held for $t_{\mathrm{step\,hold}} = 100\unit{s}$}
    \label{subfig:tf_est_input_step2}
\endgroup
\end{minipage}
\end{example}



%%%%%%%%%%%%%%%%%%%%%%%%%%%%%%%%%%%%%%%%%%%%%%%%%%%%%%%%%%%%%%%%%%%%%%%%%%%%%%
% ============================================================================
% =================  INPUT TRANSFER ORDER: INPUT SET  ===================
\section{Input Transfer Order}\label{subsecapp:input_transfe_order}
% -------------------------------------------------------------------
\graphicspath{{./Bilder/appendix/tf_order}} 
\begin{commentnotes}
\textbf{Main Idea}
\vspace{-0.5em}
\begin{itemize}[noitemsep, topsep=0pt]
	\item The order of the \gls*{transfer_map} impacts the transfer performance. We would expect, that a transfer of lower order can still achieve good results for certain frequencies. Training data from random input trajectories is used to estimate \glspl*{input_map} of different orders between two known second-order systems. The transfer error dynamics as well as the training and test errors are evaluated for all system orders. Additionally, the training data is deteriorated by output noise and the estimation is repeated.
\end{itemize}

\textbf{Conclusions}
\vspace{-0.5em}
\begin{itemize}[noitemsep, topsep=0pt]
	\item A constant gain hardly decreases the direct transfer error. A system order of at least 2 is recommended.
	\item The best transfer is achieved for the ideal system order of $K=3$
	\item Without noise, a higher system order has no real impact.
	\item With noise, a too high system order results more often in negative transfer. 
	\item With noise, higher frequencies, not part of the training (and test) input, resulted in high transfer errors. (especially for higher transfer errors)
\end{itemize}
\end{commentnotes}
% -------------------------------------------------------------------


This section explores how the model order choice impacts the transfer error dynamics for two second-order example systems. The setup is given in detail in \exref{exapp:input_transfer_order}. The main results are described in the following. 

Given two second-order \acrshort{lti} \acrshort{siso} systems with a relative order of one, the order of the ideal transfer map from \eqref{eq:ideal_sys_order} is $K=3$.
Since both systems have the same system order and relative degree, they are structurally similar. The relative degree of the ideal transfer is $m=0$.

A trivial transfer choice is a static map with only a single scalar parameter tuned to transfer trajectories between two systems. This is equivalent to a transfer function of order $K=0$. As seen in \secref{subsec:static_map}, a static map cannot account for frequency-dependent differences in the phase and gain of the systems.
%The resulting dynamics of the transfer error are mainly impacted by the choice of the training input.  
While this transfer is easily implemented, the benefits are limited to specific inputs, and the decrease in the transfer error compared to the direct transfer is only marginal. The transfer error dynamics for $K=0$ are shown for the two example systems in \figref{subfigapp:tf_order_a}.

Increasing the order of the estimated \gls*{input_map} to $K=1$ improves the transfer performers. However, a transfer function of order one lacks the ability to model oscillating behaviour. This may be the reason why the transfer error is still very high over a large range of inputs. This is seen in \figref{subfigapp:tf_order_c}.

When the order of the estimated system is increased to $K=2$, the error dynamics show a sudden improvement over the previous order of $K=1$. This can be seen especially in the substantial decrease of the maximum error gain $h_{\infty}$ seen in \figref{subfigapp:tf_order_e}. This may be to its ability to model oscillating behaviour. 

The ideal model order of $K=3$ results in the best overall transfer performance. Under perfect conditions\footnote{no disturbances / noise affect the outputs of the training input}, the estimated transfer function is so similar to the ideal \gls*{input_map} that only an insignificant transfer error remained, even for frequencies not contained in the training data (\figref{subfigapp:tf_order_g}).

Increasing the model order further ($K>3$) does not affect the transfer error under ideal conditions. However, if output noise deteriorates the training data even slightly, outliers with very poor transfer performance become more frequent with a higher model order of the estimated \gls*{transfer_map}. 

The effect of output noise is most noteworthy in higher frequencies, which are absent in the deterministic training input. 
This may result in a large difference between training and test errors. Increasing the model order leads to a greater difference in transfer errors between the perfect training data and data corrupted with output noise. This indicates that lower-order systems are more resilient to imperfect training data than higher-order systems. 
\figref{figapp:input_transfer_order} illustrates the transfer error dynamics in the absence of output noise on the left side and in the presence of noise on the right side.


\begin{notebox}[Random trajectories]\label{noteapp:sample_random_input}
	\textbf{How random trajectories of length $N$ and magnitude $\sigma_{\vec{u}}$ are generated in this work}~\\
	\vspace{-1em}
	\begin{enumerate}[noitemsep, topsep=0pt]
		\item A sequence of random samples is drawn from a uniform distribution.
		\item The sequence is filtered using the MATLAB method \texttt{filtfilt()} with a 6th-order Butterworth filter with a cutoff frequency of $f_{\mathrm{cutoff}}$.
		\item A sequence $\vec{u}$ containing $N$ samples is selected from the filtered sequence so that the initial value $u_0$ is close to zero. 
		\item The \textit{peak-to-rms factor} (\textit{crest factor}) of $\vec{u}$ is evaluated using the MATALB method \texttt{peak2rms}, so that $\texttt{peak2rms}(\vec{u}) \leq 3$ is given. If not, a new sequence is drawn. 
		\item The sequence $\vec{u}$ is rescaled so that its standard deviation $\mathrm{std}(\vec{u})$ matches 
		$2\sigma_{\vec{u}}$. (I.e. for $\sigma_{\vec{u}}=1$, the standard deviation of the sequence $\mathrm{std}(\vec{u})=2$, which means that most of the signal lies between $-1$ and $1$)
	\end{enumerate}	
	See the file \filenamecode{fun\_sample\_random\_input.m}.	
\end{notebox}

% -------------------------------------------------------------------
\begin{commentnotes}
\begin{table}
        \renewcommand{\arraystretch}{1.3}
        \begin{tabularx}{1\textwidth}{@{}ccccccc@{}}
            \toprule
            \textbf{Order} & \phantom{a} & \multicolumn{2}{c}{\textbf{Training Error}} & \phantom{abc} & \multicolumn{2}{c}{\textbf{Test Error}} \\ \cmidrule{3-4} \cmidrule{6-7}
            $K$ && $\sigma = 0$ & $\sigma = 0.05$ && $\sigma = 0$ & $\sigma = 0.05$ \\ \midrule
            $0$ && $9.6 \cdot 10^{-1}$ & $9.6 \cdot 10^{-1}$ && $9.8 \cdot 10^{-1}$ & $9.8 \cdot 10^{-1}$ \\
            $1$ && $8.2 \cdot 10^{-1}$ & $8.2 \cdot 10^{-1}$ && $8.4 \cdot 10^{-1}$ & $8.5 \cdot 10^{-1}$ \\
            $2$ && $5.6 \cdot 10^{-2}$ & $1.7 \cdot 10^{-1}$ && $6.0 \cdot 10^{-2}$ & $1.8 \cdot 10^{-1}$\\
            $3$ && $4 \cdot 10^{-15}$ & $0.9 \cdot 10^{-1}$ && $4 \cdot 10^{-15}$ & $1.0 \cdot 10^{-1}$\\
            $4$ && $4 \cdot 10^{-15}$ & $5.6 \cdot 10^{-2}$ && $4 \cdot 10^{-15}$ & $6.2 \cdot 10^{-2}$\\
            $5$ && $3.5 \cdot 10^{-11}$ & $9.4 \cdot 10^{-1}$ && $4.1 \cdot 10^{-11}$ & $9.5 \cdot 10^{-1}$\\
            $6$ && $4.2 \cdot 10^{-5}$ & $8.1 \cdot 10^{-1}$ && $5.3 \cdot 10^{-5}$ & $8.0 \cdot 10^{-1}$\\
%            $•$ & $• \cdot 10^{-1}$ & $• \cdot 10^{-1}$\\
%\multirow{2}{*}{\textbf{Order} $K$}
            \bottomrule
        \end{tabularx}
        \caption[Test and Training Errors of Input Transfer Maps for various System Orders]{Average training and test error for different system orders $K$ of the estimated \gls*{input_map} dynamics without output noise $\sigma = 0$ and with output noise $\sigma = 0.05$ in the training data}
        \label{tab:input_transfer_transfer_order}
\end{table} 
\end{commentnotes} 
% -------------------------------------------------------------------


% ============================================================================
\begin{example}[System order of the estimated input transfer map]\label{exapp:input_transfer_order}
%\paragraph{Input Set}\label{frame:input transfer order: input set}~\\
Consider the following two second-order minimum phase \glsxtrshort{lti} \glsxtrshort{siso} systems
\begin{equation}
    F^{(1)}(s) = \frac{2s+1}{0.9s^2+1.2s+0.5} \quad
    F^{(2)}(s) = \frac{0.5s+1}{1s^2+0.5s+1}\,. \label{eqapp:example_order_change}
\end{equation}
\figref{figapp:input_tf_order_training_data_bode} shows the bode plot of the two systems and their corresponding error dynamics.

\begingroup
	\centering
	\vspace{1em}
	\captionsetup{format=plain, type=figure, labelfont={color=example_font_color, bf}, font={color=example_font_color}}
        \includegraphics[width=0.5\textwidth]{/system_order/input transfer order systems bode.pdf}
        \caption[System Order of Transfer Maps Example (Bode Plot of the Error Dynamics)]{Bode plot of systems $F^{(1)}(s)$ and $F^{(1)}(s)$ and their error dynamics $H^{(1,2)}(s)$}
		\label{figapp:input_tf_order_training_data_bode}
\endgroup

Consider multiple input trajectories 
\begin{equation}
    \vec{u}_l \in (\mathcal{U}^{(i,j)})^N \quad
    \text{for} \quad l \in [1,\ldots,20] 
% &&\text{, } |\mathcal{U}| = 20
\end{equation} 
Each input $\vec{u}_l$ is a random input trajectory of length $t=50\unit{s}$ and magnitude one which is drawn according to \noteref{note:sample_random_input}. A sampling rate of $f_s = 100\unit{Hz}$ was chosen. \figref{subfigapp:input_tf_order_training_data_in} shows some of the randomly generated input trajectories, with one trajectory highlighted. The corresponding deterministic ($\sigma = 0$) outputs of the systems $F^{(1)}$ and $F^{(2)}$ to the inputs are shown in \figref{subfigapp:input_tf_order_training_data_out}.

In a second case, a noise sequence $\vec{w}$ is added to the deterministic output trajectories, with  
\begin{equation}
    w \sim  \mathcal{N}(0,\sigma^2) \quad
    \forall w  \in \vec{w} \qquad \text{and}
    \qquad \sigma = 0.05\,.
\end{equation} 
The resulting output trajectories, including the additional noise, are depicted in \figref{subfigapp:input_tf_order_training_data_out2}.

\begin{minipage}{1\textwidth}
\begingroup
	\centering
	\vspace{1em}
	\captionsetup{format=plain, type=figure, labelfont={color=example_font_color, bf}, font={color=example_font_color}}
    \begin{subfigure}[t]{0.515\textwidth}
        \centering\captionsetup{width=.9\linewidth}
        \includegraphics[width=\textwidth]{/system_order/input transfer order input set.pdf}
        \caption{Some training inputs with one input highlighted}
        \label{subfigapp:input_tf_order_training_data_in}
    \end{subfigure}
    \hfill
    \begin{subfigure}[t]{0.495\textwidth}
        \centering\captionsetup{width=.9\linewidth}
        \includegraphics[width=\textwidth]{/system_order_noise/input transfer order output set.pdf}
        \caption{Corresponding deterministic output trajectories to the training inputs}
        \label{subfigapp:input_tf_order_training_data_out}
    \end{subfigure}
    \hfill
    \begin{subfigure}[t]{0.495\textwidth}
        \centering\captionsetup{width=.9\linewidth}
        \includegraphics[width=\textwidth]{/system_order_noise/input transfer order output set noise 0.050.pdf}
        \caption{Corresponding output trajectories to the training inputs with artificial noise}
        \label{subfigapp:input_tf_order_training_data_out2}
    \end{subfigure}
    \caption[System Order of Transfer Maps Example (Training Data)]{Training data for \gls*{input_map} estimation of different system orders}
    \label{figapp:input_tf_order_training_data}
\endgroup
\end{minipage}
 

% ============================================================================
% =================  INPUT TRANSFER ORDER: EXAMPLE SYSTEMS  ==================

For every input $\vec{u}_l$ an \gls*{input_map} $\hat{T}^{(1 \rightarrow 2)}_l(s)$ with $l = 1,\ldots, 20$ of order $K$ in the form of \eqref{eq:estimated_tf} is estimated using the corresponding outputs $\left(\vec{y}^{(1)}, \vec{y}^{(2)}\right)_{\vec{u}_l}$. Each \gls*{input_map} is estimated once using the deterministic output trajectories and once under the presence of output noise.\\ 
The system orders of the estimated \glspl*{input_map} are increased from zero to six, so that
\begin{equation}
	K = 0,\ldots,6\,.
\end{equation}
Note that the ideal order of the \gls*{input_map} for the sytems in \eqref{eqapp:example_order_change} is $K=3$.\\

    For an \gls*{input_map} $\hat{T}^{(1 \rightarrow 2)}_l(s)$, the training error $e^{(1 \rightarrow 2)}_{\vec{u}, \mathrm{NRMS}}(\vec{u}_l)$ is calculated using the training input $\vec{u}_l$.\\
    The test error $e^{(1 \rightarrow 2)}_{\vec{u}, \mathrm{NRMS}}(\vec{u}_k)$ is calculated using the remaining test inputs $\vec{u}_k$ for $k = 1,\ldots,20$ and $k\neq l$.
    
    We then average the training error and test error over all estimated \glspl*{input_map} 
    $\hat{T}^{(1 \rightarrow 2)}_l(s)$. This is shown in \figref{figapp:tf_order_error}.
    
    The resulting transfer error dynamics are shown in \figref{figapp:input_transfer_order}.

\begin{minipage}{1\textwidth}
\begingroup
	\centering
	\vspace{1em}
	\captionsetup{format=plain, type=figure, labelfont={color=example_font_color, bf}, font={color=example_font_color}}
           \includegraphics[width=0.6\textwidth]{/system_order_noise/error_over_order.pdf}
           \caption[System Order of Transfer Maps Example (Test and Training Errors)]{Mean training error and mean test error over the order of the estimated biproper transfer function with ($\sigma = 0.05$) and without ($\sigma = 0$) noise in the training trajectories}
           \label{figapp:tf_order_error}
       \endgroup
\end{minipage}

    
% ===================================================================
\begingroup
	\centering	
	\captionsetup{format=plain, type=figure, labelfont={color=example_font_color, bf}, font={color=example_font_color}}
	\begin{minipage}{1\textwidth}
	\vspace{1em}     
       \begin{subfigure}{0.495\textwidth}
           \centering
           \includegraphics[width=\textwidth]{/system_order/input transfer order random input s0 n0.000 bode.pdf}
           \caption{$K = 0$ and $\sigma = 0$}
           \label{subfigapp:tf_order_a}
       \end{subfigure}
       \hfill
       \begin{subfigure}{0.495\textwidth}
           \centering
           \includegraphics[width=\textwidth]{/system_order_noise/input transfer order random input s0 n0.050 bode.pdf}
           \caption{$K = 0$ and $\sigma = 0.05$}
           \label{subfigapp:tf_order_b}
       \end{subfigure}
       \hfill 
       
       \medskip      
       \begin{subfigure}{0.495\textwidth}
           \centering
           \includegraphics[width=\textwidth]{/system_order/input transfer order random input s1 n0.000 bode.pdf}
           \caption{$K = 1$ and $\sigma = 0$}
           \label{subfigapp:tf_order_c}
       \end{subfigure}
       \hfill 
       \begin{subfigure}{0.495\textwidth}
           \centering
           \includegraphics[width=\textwidth]{/system_order_noise/input transfer order random input s1 n0.050 bode.pdf}
           \caption{$K = 1$ and $\sigma = 0.05$}
           \label{subfigapp:tf_order_d}
       \end{subfigure}
       \hfill
       
       \medskip
       \begin{subfigure}{0.495\textwidth}
           \centering
           \includegraphics[width=\textwidth]{/system_order/input transfer order random input s2 n0.000 bode.pdf}
           \caption{$K = 2$ and $\sigma = 0$}
           \label{subfigapp:tf_order_e}
       \end{subfigure}
       \hfill
       \begin{subfigure}{0.495\textwidth}
           \centering
           \includegraphics[width=\textwidth]{/system_order_noise/input transfer order random input s2 n0.050 bode.pdf}
           \caption{$K = 2$ and $\sigma = 0.05$}
           \label{subfigapp:tf_order_f}
       \end{subfigure}
       \hfill
       
       \medskip
       \begin{subfigure}{0.495\textwidth}
           \centering
           \includegraphics[width=\textwidth]{/system_order/input transfer order random input s3 n0.000 bode.pdf}
           \caption{$K = 3$ and $\sigma = 0$}
           \label{subfigapp:tf_order_g}
       \end{subfigure}
       \hfill
       \begin{subfigure}{0.495\textwidth}
           \centering
           \includegraphics[width=\textwidth]{/system_order_noise/input transfer order random input s3 n0.050 bode.pdf}
           \caption{$K = 3$ and $\sigma = 0.05$}
           \label{subfigapp:tf_order_h}
       \end{subfigure}
       \vspace{1em}
\end{minipage}
%\end{figure}
%\clearpage
%\begin{figure}
%       \ContinuedFloat
\newpage
\begin{minipage}{1\textwidth}
\vspace{1em}
       \begin{subfigure}{0.495\textwidth}
           \centering
           \includegraphics[width=\textwidth]{/system_order/input transfer order random input s4 n0.000 bode.pdf}
           \caption{$K = 4$ and $\sigma = 0$}
           \label{subfigapp:tf_order_i}
       \end{subfigure}
       \hfill 
       \begin{subfigure}{0.495\textwidth}
           \centering
           \includegraphics[width=\textwidth]{/system_order_noise/input transfer order random input s4 n0.050 bode.pdf}
           \caption{$K = 4$ and $\sigma = 0.05$}
           \label{subfigapp:tf_order_j}
       \end{subfigure}
       \hfill 
       
       \medskip                   
       \begin{subfigure}{0.495\textwidth}
           \centering
           \includegraphics[width=\textwidth]{/system_order/input transfer order random input s5 n0.000 bode.pdf}
           \caption{$K = 5$ and $\sigma = 0$}
           \label{subfigapp:tf_order_k}
       \end{subfigure}
       \hfill
       \begin{subfigure}{0.495\textwidth}
           \centering
           \includegraphics[width=\textwidth]{/system_order_noise/input transfer order random input s5 n0.050 bode.pdf}
           \caption{$K = 5$ and $\sigma = 0.05$}
           \label{subfigapp:tf_order_l}
       \end{subfigure}
       \hfill
       
       \medskip  
       \begin{subfigure}{0.495\textwidth}
           \centering
           \includegraphics[width=\textwidth]{/system_order/input transfer order random input s6 n0.000 bode.pdf}
           \caption{$K = 6$ and $\sigma = 0$}
           \label{subfigapp:tf_order_m}
       \end{subfigure} 
       \hfill
       \begin{subfigure}{0.495\textwidth}
           \centering
           \includegraphics[width=\textwidth]{/system_order_noise/input transfer order random input s6 n0.050 bode.pdf}
           \caption{$K = 6$ and $\sigma = 0.05$}
           \label{subfigapp:tf_order_n}
       \end{subfigure} 
       %%%%%              
       \caption[System Order of Transfer Maps Example (Bode Plot of the Transfer Error Dynamics)]{Bode diagram of the error dynamics with and without the estimated \gls*{input_map} $\hat{T}^{(1 \rightarrow 2)}_{l}$ for different system orders $K$ of the input transfer dynamics}
       \label{figapp:input_transfer_order}
%\end{figure} 
\end{minipage}
\endgroup
\end{example} 




% =============================== EXAMPLE 0  ================================
% ============================================================================
\graphicspath{{./Bilder/transfer_function_estimation/system_order/example4}} 
\clearpage
\begin{example}[Example 1: System order - constant gain]\label{ex:input_transfer_order_appex0}   
% ===================================================================
\begingroup
	\centering	
	\captionsetup{format=plain, type=figure, labelfont={color=example_font_color, bf}, font={color=example_font_color}}
	\begin{minipage}{1\textwidth}
	\vspace{1em}     
       \begin{subfigure}{0.495\textwidth}
           \centering
           \includegraphics[width=\textwidth]{/det/input transfer order random input s0 n0.000 bode.pdf}
           \caption{$K = 0$ and $\sigma = 0$}
           \label{subfig:tf_order_a_appex0}
       \end{subfigure}
       \hfill
       \begin{subfigure}{0.495\textwidth}
           \centering
           \includegraphics[width=\textwidth]{/noise/input transfer order random input s0 n0.030 bode.pdf}
           \caption{$K = 0$ and $\sigma = 0.03$}
           \label{subfig:tf_order_b_appex0}
       \end{subfigure}
       \hfill 
       
       \medskip      
       \begin{subfigure}{0.495\textwidth}
           \centering
           \includegraphics[width=\textwidth]{/det/input transfer order random input s1 n0.000 bode.pdf}
           \caption{$K = 1$ and $\sigma = 0$}
           \label{subfig:tf_order_c_appex0}
       \end{subfigure}
       \hfill 
       \begin{subfigure}{0.495\textwidth}
           \centering
           \includegraphics[width=\textwidth]{/noise/input transfer order random input s1 n0.030 bode.pdf}
           \caption{$K = 1$ and $\sigma = 0.03$}
           \label{subfig:tf_order_d_appex0}
       \end{subfigure}
       \hfill
       
       \medskip
       \begin{subfigure}{0.495\textwidth}
           \centering
           \includegraphics[width=\textwidth]{/det/input transfer order random input s2 n0.000 bode.pdf}
           \caption{$K = 2$ and $\sigma = 0$}
           \label{subfig:tf_order_e_appex0}
       \end{subfigure}
       \hfill
       \begin{subfigure}{0.495\textwidth}
           \centering
           \includegraphics[width=\textwidth]{/noise/input transfer order random input s2 n0.030 bode.pdf}
           \caption{$K = 2$ and $\sigma = 0.03$}
           \label{subfig:tf_order_f_appex0}
       \end{subfigure}
       \hfill
       
       \medskip
       \begin{subfigure}{0.495\textwidth}
           \centering
           \includegraphics[width=\textwidth]{/det/input transfer order random input s3 n0.000 bode.pdf}
           \caption{$K = 3$ and $\sigma = 0$}
           \label{subfig:tf_order_g_appex0}
       \end{subfigure}
       \hfill
       \begin{subfigure}{0.495\textwidth}
           \centering
           \includegraphics[width=\textwidth]{/noise/input transfer order random input s3 n0.030 bode.pdf}
           \caption{$K = 3$ and $\sigma = 0.03$}
           \label{subfig:tf_order_h_appex0}
       \end{subfigure}
       \vspace{1em}
\end{minipage}

\newpage
\begin{minipage}{1\textwidth}
\vspace{1em}
       \begin{subfigure}{0.495\textwidth}
           \centering
           \includegraphics[width=\textwidth]{/det/input transfer order random input s4 n0.000 bode.pdf}
           \caption{$K = 4$ and $\sigma = 0$}
           \label{subfig:tf_order_i_appex0}
       \end{subfigure}
       \hfill 
       \begin{subfigure}{0.495\textwidth}
           \centering
           \includegraphics[width=\textwidth]{/noise/input transfer order random input s4 n0.030 bode.pdf}
           \caption{$K = 4$ and $\sigma = 0.03$}
           \label{subfig:tf_order_j_appex0}
       \end{subfigure}
       \hfill 
       
       \medskip                   
       \begin{subfigure}{0.495\textwidth}
           \centering
           \includegraphics[width=\textwidth]{/det/input transfer order random input s5 n0.000 bode.pdf}
           \caption{$K = 5$ and $\sigma = 0$}
           \label{subfig:tf_order_k_appex0}
       \end{subfigure}
       \hfill
       \begin{subfigure}{0.495\textwidth}
           \centering
           \includegraphics[width=\textwidth]{/noise/input transfer order random input s5 n0.030 bode.pdf}
           \caption{$K = 5$ and $\sigma = 0.03$}
           \label{subfig:tf_order_l_appex0}
       \end{subfigure}
       \hfill
       
       \medskip  
       \begin{subfigure}{0.495\textwidth}
           \centering
           \includegraphics[width=\textwidth]{/det/input transfer order random input s6 n0.000 bode.pdf}
           \caption{$K = 6$ and $\sigma = 0$}
           \label{subfig:tf_order_m_appex0}
       \end{subfigure} 
       \hfill
       \begin{subfigure}{0.495\textwidth}
           \centering
           \includegraphics[width=\textwidth]{/noise/input transfer order random input s6 n0.030 bode.pdf}
           \caption{$K = 6$ and $\sigma = 0.03$}
           \label{subfig:tf_order_n_appex0}
       \end{subfigure}        
       \hfill
       
       \medskip  
       \begin{subfigure}{0.495\textwidth}
           \centering
           \includegraphics[width=\textwidth]{/det/input transfer order random input s7 n0.000 bode.pdf}
           \caption{$K = 7$ and $\sigma = 0$}
           \label{subfig:tf_order_o_appex0}
       \end{subfigure} 
       \hfill
       \begin{subfigure}{0.495\textwidth}
           \centering
           \includegraphics[width=\textwidth]{/noise/input transfer order random input s7 n0.030 bode.pdf}
           \caption{$K = 7$ and $\sigma = 0.03$}
           \label{subfig:tf_order_p_appex0}
       \end{subfigure} 
       %%%%%              
       \caption[System Order of Transfer Maps Example (Bode Plot of the Transfer Error Dynamics)]{Bode diagram of the error dynamics with and without the estimated \gls*{input_map} $\hat{T}^{(1 \rightarrow 2)}_{l}$ for different system orders $K$ of the input transfer dynamics}
       \label{fig:input_transfer_order_appex0}
\end{minipage}
\endgroup
\end{example}  
%%%%%%%%%%%%%%%%%%%%%%%%%%%%%%%%%%%%%%%%%%%%%%%%%%%%%%%%%%%%%%%%%%%%%%%%%%%%%
\clearpage

 

% =============================== EXAMPLE 2  ================================
% ============================================================================
\graphicspath{{./Bilder/transfer_function_estimation/system_order/example2}} 
\begin{example}[Example 3: System order - complex error dynamics I]\label{ex:input_transfer_order_appex2}    
% ===================================================================
\begingroup
	\centering	
	\captionsetup{format=plain, type=figure, labelfont={color=example_font_color, bf}, font={color=example_font_color}}
	\begin{minipage}{1\textwidth}
	\vspace{1em}     
       \begin{subfigure}{0.495\textwidth}
           \centering
           \includegraphics[width=\textwidth]{/det/input transfer order random input s0 n0.000 bode.pdf}
           \caption{$K = 0$ and $\sigma = 0$}
           \label{subfig:tf_order_a_appex2}
       \end{subfigure}
       \hfill
       \begin{subfigure}{0.495\textwidth}
           \centering
           \includegraphics[width=\textwidth]{/noise/input transfer order random input s0 n0.030 bode.pdf}
           \caption{$K = 0$ and $\sigma = 0.03$}
           \label{subfig:tf_order_b_appex2}
       \end{subfigure}
       \hfill 
       
       \medskip      
       \begin{subfigure}{0.495\textwidth}
           \centering
           \includegraphics[width=\textwidth]{/det/input transfer order random input s1 n0.000 bode.pdf}
           \caption{$K = 1$ and $\sigma = 0$}
           \label{subfig:tf_order_c_appex2}
       \end{subfigure}
       \hfill 
       \begin{subfigure}{0.495\textwidth}
           \centering
           \includegraphics[width=\textwidth]{/noise/input transfer order random input s1 n0.030 bode.pdf}
           \caption{$K = 1$ and $\sigma = 0.03$}
           \label{subfig:tf_order_d_appex2}
       \end{subfigure}
       \hfill
       
       \medskip
       \begin{subfigure}{0.495\textwidth}
           \centering
           \includegraphics[width=\textwidth]{/det/input transfer order random input s2 n0.000 bode.pdf}
           \caption{$K = 2$ and $\sigma = 0$}
           \label{subfig:tf_order_e_appex2}
       \end{subfigure}
       \hfill
       \begin{subfigure}{0.495\textwidth}
           \centering
           \includegraphics[width=\textwidth]{/noise/input transfer order random input s2 n0.030 bode.pdf}
           \caption{$K = 2$ and $\sigma = 0.03$}
           \label{subfig:tf_order_f_appex2}
       \end{subfigure}
       \hfill
       
       \medskip
       \begin{subfigure}{0.495\textwidth}
           \centering
           \includegraphics[width=\textwidth]{/det/input transfer order random input s3 n0.000 bode.pdf}
           \caption{$K = 3$ and $\sigma = 0$}
           \label{subfig:tf_order_g_appex2}
       \end{subfigure}
       \hfill
       \begin{subfigure}{0.495\textwidth}
           \centering
           \includegraphics[width=\textwidth]{/noise/input transfer order random input s3 n0.030 bode.pdf}
           \caption{$K = 3$ and $\sigma = 0.03$}
           \label{subfig:tf_order_h_appex2}
       \end{subfigure}
       \vspace{1em}
\end{minipage}

\newpage
\begin{minipage}{1\textwidth}
\vspace{1em}
       \begin{subfigure}{0.495\textwidth}
           \centering
           \includegraphics[width=\textwidth]{/det/input transfer order random input s4 n0.000 bode.pdf}
           \caption{$K = 4$ and $\sigma = 0$}
           \label{subfig:tf_order_i_appex2}
       \end{subfigure}
       \hfill 
       \begin{subfigure}{0.495\textwidth}
           \centering
           \includegraphics[width=\textwidth]{/noise/input transfer order random input s4 n0.030 bode.pdf}
           \caption{$K = 4$ and $\sigma = 0.03$}
           \label{subfig:tf_order_j_appex2}
       \end{subfigure}
       \hfill 
       
       \medskip                   
       \begin{subfigure}{0.495\textwidth}
           \centering
           \includegraphics[width=\textwidth]{/det/input transfer order random input s5 n0.000 bode.pdf}
           \caption{$K = 5$ and $\sigma = 0$}
           \label{subfig:tf_order_k_appex2}
       \end{subfigure}
       \hfill
       \begin{subfigure}{0.495\textwidth}
           \centering
           \includegraphics[width=\textwidth]{/noise/input transfer order random input s5 n0.030 bode.pdf}
           \caption{$K = 5$ and $\sigma = 0.03$}
           \label{subfig:tf_order_l_appex2}
       \end{subfigure}
       \hfill
       
       \medskip  
       \begin{subfigure}{0.495\textwidth}
           \centering
           \includegraphics[width=\textwidth]{/det/input transfer order random input s6 n0.000 bode.pdf}
           \caption{$K = 6$ and $\sigma = 0$}
           \label{subfig:tf_order_m_appex2}
       \end{subfigure} 
       \hfill
       \begin{subfigure}{0.495\textwidth}
           \centering
           \includegraphics[width=\textwidth]{/noise/input transfer order random input s6 n0.030 bode.pdf}
           \caption{$K = 6$ and $\sigma = 0.03$}
           \label{subfig:tf_order_n_appex2}
       \end{subfigure}        
       \hfill
       
       \medskip  
       \begin{subfigure}{0.495\textwidth}
           \centering
           \includegraphics[width=\textwidth]{/det/input transfer order random input s7 n0.000 bode.pdf}
           \caption{$K = 7$ and $\sigma = 0$}
           \label{subfig:tf_order_o_appex2}
       \end{subfigure} 
       \hfill
       \begin{subfigure}{0.495\textwidth}
           \centering
           \includegraphics[width=\textwidth]{/noise/input transfer order random input s7 n0.030 bode.pdf}
           \caption{$K = 7$ and $\sigma = 0.03$}
           \label{subfig:tf_order_p_appex2}
       \end{subfigure} 
       %%%%%              
       \caption[System Order of Transfer Maps Example (Bode Plot of the Transfer Error Dynamics)]{Bode diagram of the error dynamics with and without the estimated \gls*{input_map} $\hat{T}^{(1 \rightarrow 2)}_{l}$ for different system orders $K$ of the input transfer dynamics}
       \label{fig:input_transfer_order_appex2}
\end{minipage}
\endgroup

\begin{minipage}{1\textwidth}
\begingroup
	\centering
	\vspace{1em}
	\captionsetup{format=plain, type=figure, labelfont={color=example_font_color, bf}, font={color=example_font_color}}
    \begin{subfigure}[t]{0.32\textwidth}
        \centering\captionsetup{width=.9\linewidth}
        \includegraphics[width=\textwidth]{/noise/input transfer order input set.pdf}
        \caption{Some training inputs with one input highlighted}
        \label{subfig:input_tf_order_training_data_in_appex2}
    \end{subfigure}
    \hfill
    \begin{subfigure}[t]{0.32\textwidth}
        \centering\captionsetup{width=.9\linewidth}
        \includegraphics[width=\textwidth]{/noise/input transfer order output set.pdf}
        \caption{Corresponding deterministic output trajectories to the training inputs}
        \label{subfig:input_tf_order_training_data_out_appex2}
    \end{subfigure}
    \hfill
    \begin{subfigure}[t]{0.32\textwidth}
        \centering\captionsetup{width=.9\linewidth}
        \includegraphics[width=\textwidth]{/noise/input transfer order output set noise 0.030.pdf}
        \caption{Corresponding output trajectories to the training inputs with artificial noise}
        \label{subfig:input_tf_order_training_data_out2_appex2}
    \end{subfigure}
    \caption[System Order of Transfer Maps Example (Training Data)]{Training data for \gls*{input_map} estimation of different system orders}
    \label{fig:input_tf_order_training_data_appex2}
\endgroup
\end{minipage}
\end{example}  
%%%%%%%%%%%%%%%%%%%%%%%%%%%%%%%%%%%%%%%%%%%%%%%%%%%%%%%%%%%%%%%%%%%%%%%%%%%%%



% =============================== EXAMPLE 3 ================================
% ============================================================================
\graphicspath{{./Bilder/transfer_function_estimation/system_order/example3}} 
\clearpage
\begin{example}[Example 3: System order - complex error dynamics II]\label{ex:input_transfer_order_appex3}  
\begingroup
	\centering	
	\captionsetup{format=plain, type=figure, labelfont={color=example_font_color, bf}, font={color=example_font_color}}
	\begin{minipage}{1\textwidth}
	\vspace{1em}     
       \begin{subfigure}{0.495\textwidth}
           \centering
           \includegraphics[width=\textwidth]{/det/input transfer order random input s0 n0.000 bode.pdf}
           \caption{$K = 0$ and $\sigma = 0$}
           \label{subfig:tf_order_a_appex3}
       \end{subfigure}
       \hfill
       \begin{subfigure}{0.495\textwidth}
           \centering
           \includegraphics[width=\textwidth]{/noise/input transfer order random input s0 n0.030 bode.pdf}
           \caption{$K = 0$ and $\sigma = 0.03$}
           \label{subfig:tf_order_b_appex3}
       \end{subfigure}
       \hfill 
       
       \medskip      
       \begin{subfigure}{0.495\textwidth}
           \centering
           \includegraphics[width=\textwidth]{/det/input transfer order random input s1 n0.000 bode.pdf}
           \caption{$K = 1$ and $\sigma = 0$}
           \label{subfig:tf_order_c_appex3}
       \end{subfigure}
       \hfill 
       \begin{subfigure}{0.495\textwidth}
           \centering
           \includegraphics[width=\textwidth]{/noise/input transfer order random input s1 n0.030 bode.pdf}
           \caption{$K = 1$ and $\sigma = 0.03$}
           \label{subfig:tf_order_d_appex3}
       \end{subfigure}
       \hfill
       
       \medskip
       \begin{subfigure}{0.495\textwidth}
           \centering
           \includegraphics[width=\textwidth]{/det/input transfer order random input s2 n0.000 bode.pdf}
           \caption{$K = 2$ and $\sigma = 0$}
           \label{subfig:tf_order_e_appex3}
       \end{subfigure}
       \hfill
       \begin{subfigure}{0.495\textwidth}
           \centering
           \includegraphics[width=\textwidth]{/noise/input transfer order random input s2 n0.030 bode.pdf}
           \caption{$K = 2$ and $\sigma = 0.03$}
           \label{subfig:tf_order_f_appex3}
       \end{subfigure}
       \hfill
       
       \medskip
       \begin{subfigure}{0.495\textwidth}
           \centering
           \includegraphics[width=\textwidth]{/det/input transfer order random input s3 n0.000 bode.pdf}
           \caption{$K = 3$ and $\sigma = 0$}
           \label{subfig:tf_order_g_appex3}
       \end{subfigure}
       \hfill
       \begin{subfigure}{0.495\textwidth}
           \centering
           \includegraphics[width=\textwidth]{/noise/input transfer order random input s3 n0.030 bode.pdf}
           \caption{$K = 3$ and $\sigma = 0.03$}
           \label{subfig:tf_order_h_appex3}
       \end{subfigure}
       \vspace{1em}
\end{minipage}

\newpage
\begin{minipage}{1\textwidth}
\vspace{1em}
       \begin{subfigure}{0.495\textwidth}
           \centering
           \includegraphics[width=\textwidth]{/det/input transfer order random input s4 n0.000 bode.pdf}
           \caption{$K = 4$ and $\sigma = 0$}
           \label{subfig:tf_order_i_appex3}
       \end{subfigure}
       \hfill 
       \begin{subfigure}{0.495\textwidth}
           \centering
           \includegraphics[width=\textwidth]{/noise/input transfer order random input s4 n0.030 bode.pdf}
           \caption{$K = 4$ and $\sigma = 0.03$}
           \label{subfig:tf_order_j_appex3}
       \end{subfigure}
       \hfill 
       
       \medskip                   
       \begin{subfigure}{0.495\textwidth}
           \centering
           \includegraphics[width=\textwidth]{/det/input transfer order random input s5 n0.000 bode.pdf}
           \caption{$K = 5$ and $\sigma = 0$}
           \label{subfig:tf_order_k_appex3}
       \end{subfigure}
       \hfill
       \begin{subfigure}{0.495\textwidth}
           \centering
           \includegraphics[width=\textwidth]{/noise/input transfer order random input s5 n0.030 bode.pdf}
           \caption{$K = 5$ and $\sigma = 0.03$}
           \label{subfig:tf_order_l_appex3}
       \end{subfigure}
       \hfill
       
       \medskip  
       \begin{subfigure}{0.495\textwidth}
           \centering
           \includegraphics[width=\textwidth]{/det/input transfer order random input s6 n0.000 bode.pdf}
           \caption{$K = 6$ and $\sigma = 0$}
           \label{subfig:tf_order_m_appex3}
       \end{subfigure} 
       \hfill
       \begin{subfigure}{0.495\textwidth}
           \centering
           \includegraphics[width=\textwidth]{/noise/input transfer order random input s6 n0.030 bode.pdf}
           \caption{$K = 6$ and $\sigma = 0.03$}
           \label{subfig:tf_order_n_appex3}
       \end{subfigure}        
       \hfill
       
       \medskip  
       \begin{subfigure}{0.495\textwidth}
           \centering
           \includegraphics[width=\textwidth]{/det/input transfer order random input s7 n0.000 bode.pdf}
           \caption{$K = 7$ and $\sigma = 0$}
           \label{subfig:tf_order_o_appex3}
       \end{subfigure} 
       \hfill
       \begin{subfigure}{0.495\textwidth}
           \centering
           \includegraphics[width=\textwidth]{/noise/input transfer order random input s7 n0.030 bode.pdf}
           \caption{$K = 7$ and $\sigma = 0.03$}
           \label{subfig:tf_order_p_appex3}
       \end{subfigure} 
       %%%%%              
       \caption[System Order of Transfer Maps Example (Bode Plot of the Transfer Error Dynamics)]{Bode diagram of the error dynamics with and without the estimated \gls*{input_map} $\hat{T}^{(1 \rightarrow 2)}_{l}$ for different system orders $K$ of the input transfer dynamics}
       \label{fig:input_transfer_order_appex3}
\end{minipage}
\endgroup

\begin{minipage}{1\textwidth}
\begingroup
	\centering
	\vspace{1em}
	\captionsetup{format=plain, type=figure, labelfont={color=example_font_color, bf}, font={color=example_font_color}}
    \begin{subfigure}[t]{0.32\textwidth}
        \centering\captionsetup{width=.9\linewidth}
        \includegraphics[width=\textwidth]{/noise/input transfer order input set.pdf}
        \caption{Some training inputs with one input highlighted}
        \label{subfig:input_tf_order_training_data_in_appex1}
    \end{subfigure}
    \hfill
    \begin{subfigure}[t]{0.32\textwidth}
        \centering\captionsetup{width=.9\linewidth}
        \includegraphics[width=\textwidth]{/noise/input transfer order output set.pdf}
        \caption{Corresponding deterministic output trajectories to the training inputs}
        \label{subfig:input_tf_order_training_data_out_appex1}
    \end{subfigure}
    \hfill
    \begin{subfigure}[t]{0.32\textwidth}
        \centering\captionsetup{width=.9\linewidth}
        \includegraphics[width=\textwidth]{/noise/input transfer order output set noise 0.030.pdf}
        \caption{Corresponding output trajectories to the training inputs with artificial noise}
        \label{subfig:input_tf_order_training_data_out2_appex1}
    \end{subfigure}
    \caption[System Order of Transfer Maps Example (Training Data)]{Training data for \gls*{input_map} estimation of different system orders}
    \label{fig:input_tf_order_training_data_appex1}
\endgroup
\end{minipage}
\end{example}  
%%%%%%%%%%%%%%%%%%%%%%%%%%%%%%%%%%%%%%%%%%%%%%%%%%%%%%%%%%%%%%%%%%%%%%%%%%%%%

\chapter{Simulation}
% ____________________________________________________________________
% ======================== Spring Damper System =====================
\section{Double Oscillator}\label{sec:app_sim1}
\graphicspath{{./Bilder/simulation_linear}} 
% ====================================================================

\subsection{Derivation of the Transfer Function}\label{sec:derivation_double_osci}
The dynamical system used in the simulation \secref{sec:double_oscillator} is derived from the state-space system:
\begin{equation}
	\vec{x} = 
		\begin{bmatrix}
			x_1 & \dot{x}_1 & x_2 & \dot{x}_2		
		\end{bmatrix}^\mathrm{T}
		\qquad u = F \qquad y = x_1
\end{equation}
\begin{align}
	\dot{\vec{x}} &= 
%			\begin{bmatrix}
%				0 			& 1 		& 0 			& 0			\\
%				-c_1/m_1	& -d_1/m_1  & c_2/m_1 		& d_2/m_1	\\
%				0			& 0			& 1				& 0			\\
%				c_1/m_2		& d_1/m_2  	& (c_1+c_2)/m_2 & (d_1+d_2)/m_2
%			\end{bmatrix} \vec{x} + 
		\begin{bmatrix}
			0 			& 1 		& 0 			& 0			\\
			-\frac{c_1}{m_1}	& -\frac{d_1}{m_1}  & \frac{c_2}{m_1} 		& \frac{d_2}{m_1}	\\
			0			& 0			& 1				& 0			\\
			\frac{c_1}{m_2}		& \frac{d_1}{m_2}  	& -\frac{c_1+c_2}{m_2} & -\frac{d_1+d_2}{m_2}
		\end{bmatrix} \vec{x} + 
		\begin{bmatrix}
			0 \\ \frac{1}{m_1} \\ 0 \\ 0		
		\end{bmatrix} u 	\\		
	y &= 
		\begin{bmatrix}
			1 & 0 & 0 & 0		
		\end{bmatrix} \vec{x} + [0] u	
\end{align}
The derivatives $\dot{\vec{x}}$ is obtained from the internal and external forces on the masses $m_1$ and $m_2$ visualized in \figref{fig:app_sim1_forces}. The transfer function description \eqref{eq:sim1_dynamics} used in the simulation is than given as
\begin{equation}
	F(s) = \mathrm{C}\left(s\mathrm{I}-\mathrm{A}\right)^{-1}\mathrm{B}\,.
\end{equation}

\begin{figure}
        \centering
        \includegraphics[width=0.30\textwidth]{double_oscillator/double_oscillator_freischnitt.pdf}
        \caption[Double Oscillator -- External and Internal Forces]{External and internal forces on mass $m_1$ and $m_2$ for the derivation of the state-space model}
        \label{fig:app_sim1_forces}
\end{figure}
	
\subsection{Distribution of Simulated Systems}
\figref{fig:sim1_app_param_dist} shows the underlying distribution from which the parameters of the dynamical systems are drawn and how the sampled parameters actually distribute. Recall that $L=100$ pairs of systems were simulated. Since each system contains two parameters of each type (mass, damping constant and spring constant) and there are two systems in each pair, a total of $4L$ parameters is drawn from each distribution.\\
%\figref{fig:sim1_app_pzmap} shows the poles and zeros of the drawn systems. Note that all systems are minimum phase. One pair if systems is highlighted. The highlighted pair is used in figures \figref{} as an example system to demonstrate the approach.
	
\begin{figure}
    \centering
    \begin{subfigure}[t]{0.495\textwidth}
        \centering
        \includegraphics[width=\textwidth]{double_oscillator/Sim1 tf random Parameter Distribution Mass m.pdf}
        \caption{Masses $m_1, m_2$}
    \end{subfigure}
    \hfill
    \begin{subfigure}[t]{0.495\textwidth}
        \centering
        \includegraphics[width=\textwidth]{double_oscillator/Sim1 tf random Parameter Distribution Damping Constant d.pdf}
        \caption{Damping constants $d_1, d_2$}
    \end{subfigure}
    \hfill
    \begin{subfigure}[t]{0.495\textwidth}
        \centering
        \includegraphics[width=\textwidth]{double_oscillator/Sim1 tf random Parameter Distribution Spring Constant c.pdf}
        \caption{Spring constants $c_1, c_2$}
    \end{subfigure}
    \caption[Double Oscillator -- Parameter Distributions]{Underlying distribution and actual sampled parameters of the simulated spring-damper systems}
    \label{fig:sim1_app_param_dist}
\end{figure}	
	
%\begin{figure}
%        \centering
%        \includegraphics[width=0.5\textwidth]{double_oscillator/Sim1 tf random Parameter Distribution PZ Map2.pdf}
%        \caption{Appendix: Pole-Zero map of randomly drawn double mass spring-damper systems. The highlighted pair of systems $\left(F^{(1)}(s), F^{(2)}(s)\right)_k$ is used as the example system in ...(insert image links)}
%        \label{fig:sim1_app_pzmap}
%\end{figure}	
\subsection{Further Results}\label{sec:sim1_further_results}
    \begin{figure}[h!]
        \centering
        \includegraphics[width=1\textwidth]{double_oscillator/Sim1 tf random BAR_Order_training.pdf}
        \caption[Double Oscillator -- Transfer Error (Transfer Function Training Error)]{Normalized transfer error for the transfer of chirp training trajectories (\textit{training error}) using transfer functions of different system orders. Results are similar to the transfer of random test trajectories.}
        \label{fig:sim1_rand_tf_order_train}
    \end{figure}

\subsection{Pure Sine Test Trajectories}\label{subsec:app_sim1_sine}
 Instead of evaluating the input transfer on random input sequences, pure sine-wave input signals of different frequencies $f_{\mathrm{sine}}$ are used to validate the estimated input-transfer methods $\hat T^{(1 \rightarrow 2)}$ and $\vec{\hat T}^{(1 \rightarrow 2)}$. 
Evaluated frequencies are
\begin{equation}
	f_{\mathrm{sine}} \in [0.03,\; 0.05,\; 0.08,\; 0.13,\; 0.20,\; 0.32,\; 0.50,\; 0.80,\; 1.25]\unit{Hz}\,.
\end{equation}

\figref{fig:sim1_sine} shows the input signals for  $f_{\mathrm{sine}}=0.05\unit{Hz}$ and $f_{\mathrm{sine}}=0.20\unit{Hz}$. Additionally, the figure shows the outputs with and without input transformation for a selected pair of test systems 
$\left(F^{(1)}(s), F^{(2)}(s)\right)_l$. Input transfer was done using a transfer function of order 2. %and a lifted system matrix with $0.5N$ parameters.


\begin{figure}
    \centering
    \begin{subfigure}[t]{0.495\textwidth}
        \centering
        \includegraphics[width=\textwidth]{double_oscillator/Sim1 tf sine Test Inputs 0.05.pdf}
        \caption{Sine input of $f_{\mathrm{sine}}=0.05\unit{Hz}$}
    \end{subfigure}
    \hfill
    \begin{subfigure}[t]{0.495\textwidth}
        \centering
        \includegraphics[width=\textwidth]{double_oscillator/Sim1 tf sine Test Inputs 0.20.pdf}
         \caption{Sine input of $f_{\mathrm{sine}}=0.20\unit{Hz}$}
    \end{subfigure}
	%%%
	\begin{subfigure}[t]{0.495\textwidth}
        \centering
        \includegraphics[width=\textwidth]{double_oscillator/Sim1 tf sine Test Outputs with Transfer 0.05.pdf}
        \caption{Input-transfer for $f_{\mathrm{sine}}=0.05\unit{Hz}$ using a transfer function of order 2}
    \end{subfigure}
    \hfill
    \begin{subfigure}[t]{0.495\textwidth}
        \centering
        \includegraphics[width=\textwidth]{double_oscillator/Sim1 tf sine Test Outputs with Transfer 0.20.pdf}
         \caption{Input-transfer for $f_{\mathrm{sine}}=0.20\unit{Hz}$ using a transfer function of order 2}
    \end{subfigure}    
	%%%
%	\begin{subfigure}[t]{0.495\textwidth}
%        \centering
%        \includegraphics[width=\textwidth]{double_oscillator/Sim1 lift sine Test Outputs with Transfer 0.05.pdf}
%        \caption{Input-transfer for $f_{\mathrm{sine}}=0.05\unit{Hz}$ using a lifted system with $0.5N$ parameters}
%    \end{subfigure}
%    \hfill
%    \begin{subfigure}[t]{0.495\textwidth}
%        \centering
%        \includegraphics[width=\textwidth]{double_oscillator/Sim1 lift sine Test Inputs 0.20.pdf}
%        \caption{Input-transfer for $f_{\mathrm{sine}}=0.20\unit{Hz}$ using a lifted system with $0.5N$ parameters}
%    \end{subfigure}     
%    %%%%
    \caption[Double Oscillator -- Input Trajectories and Output Error Trajectories (Sinusoidal Input)]{Sinewave test inputs and their corresponding input-transfer results using a transfer-function}
    \label{fig:sim1_sine}
\end{figure}



\figref{fig:sim1_sine_results} shows the average normalized transfer error for different test frequencies $f_{\mathrm{sine}}$ for an estimated transfer functions. The general results are similar to those regarding the order of the estimated transfer, with $K=0$ being only slightly beneficial compared to the direct transfer error and $K=2$ resulting in the best overall transfer performance with low mean and standard deviation of the normalized transfer error, especially when output noise is considered. On average, positive transfer can be achieved in most cases as long as the input-transfer is not a constant gain ($K\neq0$).

The best input transfer is achieved for test inputs with a frequency of around $f_{\mathrm{sine}}= 0.13\unit{Hz}$. The training input might excite the systems very well for that frequency, while other frequencies are less prevalent in the training input. This is most notable for $K=0$ where the normalized transfer significantly decreases at first but than increases again for higher frequencies. This is seen in both cases, with and without output noise.\\
Higher transfer function orders of $K\geq6$ result in a greater normalized transfer error for higher frequencies when noise is considered in \figref{subfig:sim1_sine_noise}, but less in the deterministic case of \figref{subfig:sim1_sine_det}. 
This might be due to over-adaptation of transfer functions with greater order to the high frequency components in the noise.

In addition to the previous results, the training error is depicted on the far left side of the scale. As expected, the training error is in general smaller than the test errors. The behaviour of the training error is very similar to the test error regarding the order of the estimated transfer-function. This suggests, that the estimated input-transfer generalises very well with other inputs, but its over all performance depends heavily on the system order.\\
 
\begin{figure}
    \centering
    \begin{subfigure}[t]{1\textwidth}
        \centering
        \includegraphics[width=\textwidth]{double_oscillator/Sim1 tf sine BAR Cutoff x Order noise0.0000.pdf}
        \caption{Normalised Transfer Error for estimated input-transfer without output noise}
        \label{subfig:sim1_sine_det}
    \end{subfigure}
    \hfill
    \begin{subfigure}[t]{1\textwidth}
        \centering
        \includegraphics[width=\textwidth]{double_oscillator/Sim1 tf sine BAR Cutoff x Order noise0.0200.pdf}
        \caption{Normalised Transfer Error for estimated input-transfer without output noise}
        \label{subfig:sim1_sine_noise}
    \end{subfigure}
    \hfill
    \begin{subfigure}[t]{1\textwidth}
        \centering
        \includegraphics[width=\textwidth]{double_oscillator/Sim1 tf sine NaN.pdf}
        \caption{Percentage of attempted estimation which failed or were removed as outliers}
        \label{subfig:sim1_sine_failes}
    \end{subfigure}
    \caption[Double Oscillator -- Results for Sinusoidal Test Inputs (Transfer Function)]{Input-transfer using an estimated transfer function of different orders for various test frequencies}
    \label{fig:sim1_sine_results}
\end{figure}

\figref{fig:sim1_sine_results_lsd} shows the average normalized transfer error for different test frequencies $f_{\mathrm{sine}}$ for an estimated lifted system input-transfer. For both cases, with and without output noise, the input transfer results in negative transfer for higher test frequencies, thus degrading the initial performance of the direct transfer. In the deterministic case (without output noise) seen in \figref{subfig:sim1_sine_det_lsd}, positive transfer can be achieved on average for higher frequencies only when $1N$ parameters are estimated. 
Reducing the parameter limit for higher frequencies greatly increases the transfer error.\\
The opposite seems true when output noise is considered, seen in \figref{subfig:sim1_sine_noise_lsd}. Fewer estimated parameters result in a lower transfer error for high frequencies. However, the transfer is still negative.\\ 
This shows, that the performance of the lifted system input transfer depends much more heavily on the frequency of the test data than the input transfer using a transfer function, where positive transfer was achieved in most cases. 

%\paragraph{Lifted System Estimation: Transfer Error (deterministic)}~\\
%\begin{figure}
%        \centering
%        \includegraphics[width=0.96\textwidth]{double_oscillator/Sim1 lift sine BAR Cutoff x Parameter Limit noise0.0000.pdf}
%\end{figure}
%
%
%
%\paragraph{Lifted System Estimation: Transfer Error (noisy)}~\\
%\begin{figure}
%        \centering
%        \includegraphics[width=0.96\textwidth]{double_oscillator/Sim1 lift sine BAR Cutoff x Parameter Limit noise0.0200.pdf}
%\end{figure}
%
%
%\paragraph{Lifted System Estimation: Failed Estimations}~\\
%\begin{figure}
%        \centering
%        \includegraphics[width=0.96\textwidth]{double_oscillator/Sim1 lift sine NaN.pdf}
%\end{figure}


\begin{figure}
    \centering
    \begin{subfigure}[t]{1\textwidth}
        \centering
        \includegraphics[width=\textwidth]{double_oscillator/Sim1 lift sine BAR Cutoff x Parameter Limit noise0.0000.pdf}
        \caption{Normalised Transfer Error for estimated input-transfer without output noise}
        \label{subfig:sim1_sine_det_lsd}
    \end{subfigure}
    \hfill
    \begin{subfigure}[t]{1\textwidth}
        \centering
        \includegraphics[width=\textwidth]{double_oscillator/Sim1 lift sine BAR Cutoff x Parameter Limit noise0.0200.pdf}
        \caption{Normalised Transfer Error for estimated input-transfer without output noise}
        \label{subfig:sim1_sine_noise_lsd}
    \end{subfigure}
    \hfill
    \begin{subfigure}[t]{1\textwidth}
        \centering
        \includegraphics[width=\textwidth]{double_oscillator/Sim1 lift sine NaN.pdf}
        \caption{Percentage of attempted estimation which failed or were removed as outliers}
        \label{subfig:sim1_sine_failes_lsd}
    \end{subfigure}
    \caption[Double Oscillator -- Results for Sinusoidal Test Inputs (Transfer Matrix)]{Input-transfer using an estimated lifted system matrix of different orders for various test frequencies for double oscillator systems}
    \label{fig:sim1_sine_results_lsd}
\end{figure}


% ____________________________________________________________________
% ======================== Pendulum on cart =====================
\section{Inverted Pendulum on a Cart}\label{sec:app_sim2}
\graphicspath{{./Bilder/simulation_linear}} 
% ====================================================================
\subsection{Dynamics of the Linearised IPC}
Linearising the pendulum dynamics for $\theta = \pi $ results in the following state-space system
\begin{equation}
	\vec{x} = 
		\begin{bmatrix}
			x & \dot{x} & \varphi & \dot{\varphi}		
		\end{bmatrix}^\mathrm{T}
		\qquad u = F \qquad 
		y = \begin{bmatrix}
				x  & \varphi	
			\end{bmatrix}^\mathrm{T}
\end{equation}
\begin{align}
	\dot{\vec{x}} &= \mathbf{A} \vec{x} + \mathbf{B} u 	\\		
	y &= \mathbf{C} + [0] u	
\end{align}

\begin{align}
	p &= I(M+m)+M m l^2 \\
	\mathbf{A} &= \frac{1}{p}
		\begin{bmatrix}
			0 			& 1 		  & 0 			& 0			\\
			0			& -d(I+ml^2)  & m^2 g l^2	& 0			\\
			0			& 0			  & 0			& 1			\\
			0			& - m l d  	  & m g l (M+m) & 0
		\end{bmatrix}\\			
	\mathbf{B} &= \frac{1}{p}
		\begin{bmatrix}
			0 \\ I+m l^2 \\ 0 \\ ml		
		\end{bmatrix} 	\qquad
	\mathbf{C} = 
		\begin{bmatrix}
			1 & 0 & 0 & 0	\\
			0 & 0 & 1 & 0	
		\end{bmatrix} 			
\end{align}

%\figref{fig:sim1_app_pzmap} shows the poles and zeros of the linearised, controlled systems. Note that all systems are stable.
%\begin{figure}
%        \centering
%        \includegraphics[width=0.5\textwidth]
%        {pendulum/Sim IPC tf random linear Parameter Distribution PZ Map1.pdf}
%        \caption{?? Zeros missing Pole-zero map of all linearised inverted pendulum on a cart systems}
%        \label{fig:sim2_app_pzmap}
%\end{figure}  

Each input-transfer system is estimated for $L=100$ pairs of test systems. Five trajectories are drawn per cutoff frequency of the test inputs. For all pairs of systems and all input trajectories, the transfer error was calculated. However, since the systems can get unstable for certain inputs, some error values are very different from the majority and could be considered as outliers. To ensure, that the mean transfer error is not to much affected by outliers, values below the 3rd percentile and above the 97th percentile are removed from the dataset.  
\begin{notebox}
	Keep in mind, that the transfer error values are not normally distributed, since the error can get arbitrarily large but not smaller than zero. While the mean value should still give a good measure of what transfer error to expect, outliers with extremely large errors have a far greater effect on the mean value, than outliers with extremely small values.
\end{notebox} 

% ====================================================================
\subsection{Input Transfer in IPC Systems for Inputs of Small Magnitude}
The simulative evaluation of \secref{sec:ipc} was repeated using inputs which excite the systems mostly within a linear range. For this, the gain $\alpha_l$ in \eqref{eq:sim2_input_scaling} previously set in \secref{subsec:sim2_training_and_test_data} is reduced by an order of magnitude. Inputs are therefore scaled by
\begin{equation}
	\vec{u}_l = a_l / 10 \cdot \vec{u} \qquad \text{for} 
	\qquad l = 1,\hdots,L \,.
\end{equation}
\begin{notebox}
	This does not necessarily mean, that the systems are linear. Input-transfer can still result in inputs which produce fairly nonlinear responses.\\
	The magnitude of the output noise was not scaled to the input. The effect of the noise in this section is therefore significantly higher then in \secref{sec:ipc} (where the magnitude of the inputs is greater) as the noise-to-signal ratio is different.
\end{notebox}

Few example inputs (with one highlighted) for two different $f_{\mathrm{cutoff}}$ are shown in the upper images of \figref{fig:sim2_test_inputs_outputs}. In addition, the resulting displacement $y$ and the pendulums angle $\varphi$ of two example systems are plotted for the highlighted inputs. 

\begin{figure}
        \centering
        \includegraphics[width=0.95\textwidth]
        {pendulum/Sim IPC tf random linear Training Chirp Outputs}
        \caption[IPC -- Training Data (Low Amplitude)]{Output to the chirp input for one pair of \acrshort{ipc} systems}
\end{figure}

\begin{figure}
    \centering
    \begin{subfigure}[t]{0.495\textwidth}
        \centering
        \includegraphics[width=\textwidth]{pendulum/Sim IPC tf random linear Test Inputs 0.20.pdf}
        \caption{Test inputs for $f_{\mathrm{cutoff}}=0.2\unit{Hz}$ with one input highlighted}
        \label{subfig:sim2_test_input_slow}
    \end{subfigure}
    \hfill
    \begin{subfigure}[t]{0.495\textwidth}
        \centering
        \includegraphics[width=\textwidth]{pendulum/Sim IPC tf random linear Test Inputs 0.80.pdf}
        \caption{Test inputs for $f_{\mathrm{cutoff}}=0.8\unit{Hz}$ with one input highlighted}
        \label{subfig:sim2_test_input_fast}
    \end{subfigure}
    \hfill    
    \medskip
    %%%
    \begin{subfigure}[t]{0.495\textwidth}
        \centering
        \includegraphics[width=\textwidth]{pendulum/Sim IPC tf random linear Outputs Linear Reference Y 2.pdf}
        \caption{Displacement for a test input with $f_{\mathrm{cutoff}}=0.2\unit{Hz}$}
    \end{subfigure}
    \hfill
    \begin{subfigure}[t]{0.495\textwidth}
        \centering
        \includegraphics[width=\textwidth]{pendulum/Sim IPC tf random linear Outputs Linear Reference Y 3.pdf}
        \caption{Displacement for a test input with $f_{\mathrm{cutoff}}=0.8\unit{Hz}$}
    \end{subfigure}
    \hfill
    \medskip
    %%%
    \begin{subfigure}[t]{0.495\textwidth}
        \centering
        \includegraphics[width=\textwidth]{pendulum/Sim IPC tf random linear Outputs Linear Reference Phi 2.pdf}
        \caption{Deflection angle for a test input with $f_{\mathrm{cutoff}}=0.2\unit{Hz}$}
    \end{subfigure}
    \hfill
    \begin{subfigure}[t]{0.495\textwidth}
        \centering
        \includegraphics[width=\textwidth]{pendulum/Sim IPC tf random linear Outputs Linear Reference Phi 3.pdf}
         \caption{Deflection angle for a test input with $f_{\mathrm{cutoff}}=0.8\unit{Hz}$}
    \end{subfigure}
    
    \caption[IPC -- Test Data (Low Amplitude)]{Example test inputs and corresponding system response of two nonlinear \acrshort{ipc} systems and their linearisation for inputs of small magnitude}
    \label{fig:sim2_test_inputs_outputs}
\end{figure}


% _____________________________________________________________________________
% =========================== INPUT TRANSFER RESULTS ==========================
\figref{fig:sim2_example_outputs} shows the system response to the training input and the highlighted test inputs in \figref{subfig:sim2_test_input_slow}, \figref{subfig:sim2_test_input_fast} with and without the input-transfer for a selected pair of test systems. On the left, the input-transfer is shown for a second order transfer function. On the right, a lifted system matrix with $0.5N$ parameters is used as the input-transfer.


% EXAMPLE TF RESULTS 
\begin{figure}
    \centering
    % TRAINING
    \begin{subfigure}[t]{0.495\textwidth}
        \centering
        \includegraphics[width=\textwidth]{pendulum/Sim IPC tf random linear Training Chirp Outputs with Transfer.pdf}
        \caption{Input-transfer of the training input using a second order transfer function}
    \end{subfigure}
    \hfill
    \begin{subfigure}[t]{0.495\textwidth}
        \centering
        \includegraphics[width=\textwidth]{pendulum/Sim IPC lifted random linear Training Chirp Outputs with Transfer.pdf}
        \caption{Input-transfer of the training input using a lifted-system matrix with $0.5\cdot N$ parameters}
    \end{subfigure}
    \hfill
    \medskip
    %TEST SLOW
    \begin{subfigure}[t]{0.495\textwidth}
        \centering
        \includegraphics[width=\textwidth]{pendulum/Sim IPC tf random linear Test Outputs with Transfer 0.20.pdf}
        \caption{Input-transfer of a test input with $f_{\mathrm{cutoff}}=0.2\unit{Hz}$ using a second order transfer function}
    \end{subfigure}
    \hfill
    \begin{subfigure}[t]{0.495\textwidth}
        \centering
        \includegraphics[width=\textwidth]{pendulum/Sim IPC lifted random linear Test Outputs with Transfer 0.20.pdf}
        \caption{Input-transfer of a test input with $f_{\mathrm{cutoff}}=0.2\unit{Hz}$  using a lifted-system matrix with $0.5\cdot N$ parameters}
    \end{subfigure}
    \hfill
    \medskip
    %TEST FAST
    \begin{subfigure}[t]{0.495\textwidth}
        \centering
        \includegraphics[width=\textwidth]{pendulum/Sim IPC tf random linear Test Outputs with Transfer 0.80.pdf}
        \caption{Input-transfer of a test input with $f_{\mathrm{cutoff}}=0.8\unit{Hz}$ using a second order transfer function}
    \end{subfigure}
    \hfill
    \begin{subfigure}[t]{0.495\textwidth}
        \centering
        \includegraphics[width=\textwidth]{pendulum/Sim IPC lifted random linear Test Outputs with Transfer 0.80.pdf}
        \caption{Input-transfer of a test input with $f_{\mathrm{cutoff}}=0.8\unit{Hz}$  using a lifted-system matrix with $0.5\cdot N$ parameters}
    \end{subfigure}
    %%%%
    \caption[IPC -- Output Error Trajectories with Input Transfer (Low Amplitude)]{Input-transfer using a second order transfer function for the training and test inputs for one selected pair of test system and small inputs}
    \label{fig:sim2_example_outputs}
\end{figure}
	

% TRANSFER ERROR
\begin{figure}
    \centering
    % TF
    \begin{subfigure}[t]{0.495\textwidth}
        \centering
        \hyperlink{fig:bar_order_lift_lin}{
	        \includegraphics[width=0.94\textwidth]
	        {pendulum/Sim IPC tf random linear BAR Order.pdf}
        }
        \hypertarget{fig:bar_order_tf_lin}{}
        \caption{Input-transfer with a transfer-function of order 0 and 2 with and without output noise}
    \end{subfigure}
    \hfill
    \begin{subfigure}[t]{0.495\textwidth}
        \centering
        \hyperlink{fig:bar_freq_tf_nonlin}{
	        \includegraphics[width=0.95\textwidth]
	        {pendulum/Sim IPC tf random linear BAR Cutoff x Order noise0.0000.pdf}
        }
        \hypertarget{fig:bar_freq_tf_lin}{}
        \caption{Input-transfer with a transfer-function of order 0 and 2 without noise for different test inputs}
    \end{subfigure}
    \hfill
    \medskip
    % LIFTED
    \begin{subfigure}[t]{0.495\textwidth}
        \centering
        \hyperlink{fig:bar_order_tf_lin}{
	        \includegraphics[width=0.92\textwidth]
	        {pendulum/Sim IPC lifted random linear BAR Order.pdf}
	        }
	    \hypertarget{fig:bar_order_lift_lin}{}
        \caption{Input-transfer with a lifted-system matrix of $1N$ and $0.5N$ parameters with and without output noise}
    \end{subfigure}
    \hfill
    \begin{subfigure}[t]{0.495\textwidth}
        \centering
        \hyperlink{fig:bar_freq_lift_nonlin}{
	        \includegraphics[width=0.92\textwidth]
	        {pendulum/Sim IPC lifted random linear BAR Cutoff x Order noise0.0000.pdf}
	        }
	   	\hypertarget{fig:bar_freq_lift_lin}{}
        \caption{Input-transfer with a lifted-system matrix of $1N$ and $0.5N$ parameters without noise for different test inputs}
    \end{subfigure}
    % CAPTION
    \caption[IPC -- Transfer Error (Low Amplitude)]{Normalized Transfer Error between \acrshort{ipc} systems using an estimated transfer-function (upper figures) or estimated lifted-system matrix (lower figures) for small inputs}
    \label{fig:sim2_results}
\end{figure}



% FAILED ESTIMATIONS
\begin{figure}
    \centering
    % TF
    \begin{subfigure}[t]{0.495\textwidth}
        \centering
	        \includegraphics[width=0.94\textwidth]{pendulum/Sim IPC tf random linear NaN Order.pdf}
        \caption{Input-transfer with a transfer-function of order 0 and 2 with and without output noise}
    \end{subfigure}
    \hfill
    \begin{subfigure}[t]{0.495\textwidth}
        \centering
	        \includegraphics[width=0.95\textwidth]{pendulum/Sim IPC tf random linear NaN F_cutoff.pdf}
        \caption{Input-transfer with a transfer-function of order 0 and 2 for different test inputs}
    \end{subfigure}
    \hfill
    \medskip
    % LIFTED
    \begin{subfigure}[t]{0.495\textwidth}
        \centering
	        \includegraphics[width=0.92\textwidth]{pendulum/Sim IPC lifted random linear NaN Order.pdf}
        \caption{Input-transfer with a lifted-system matrix of $1N$ and $0.5N$ parameters with and without output noise}
    \end{subfigure}
    \hfill
    \begin{subfigure}[t]{0.495\textwidth}
        \centering
	        \includegraphics[width=0.92\textwidth]{pendulum/Sim IPC lifted random linear NaN F_cutoff.pdf}
        \caption{Input-transfer with a lifted-system matrix of $1N$ and $0.5N$ parameters for different test inputs}
    \end{subfigure}
    % CAPTION
    \caption[IPC -- Unsuccessful Trajectory Transfer (Low Amplitude)] {Percentage of attempted estimations which failed or were removed as outliers between \acrshort{ipc} systems using an estimated transfer-function (upper figures) or estimated lifted-system matrix (lower figures) for small inputs}
    \label{fig:sim2_results_nan}
\end{figure}



\chapter{Iterative Input-Transfer Estimation for Lifted Systems}\label{ch:app_iterative_input_transfer}
\graphicspath{{./Bilder/appendix/iterative_estimation}} 

The concept of iterative input-transfer estimation (\secref{sec:iterative_transfer_estiamtion}) was evaluated on two simulated \gls{ipc} systems. A unit step input was used to excite both systems and produce the reference trajectory in the first trial iteration. An estimated transfer function was used as the initial deviation system. The transfer function was then used to calculate a lifted system matrix according to \eqref{eq:ss2ls}. Two methods are evaluated. In the first method \highlight{tfest}, the initial estimation of the input transfer using the deviation system was used in all subsequent trials without further adaptation (thus, the trials were not really necessary, as the final approached input value was that of the first trial). The second method \highlight{tfest+ILC} used the proposed adaptation of the input transfer with \textit{iterative model learning}.\\
The input transfer was evaluated in both direction - with the slow system producing the reference output and with the quicker system producing the reference output. The initial response to the step input is shown in \figref{fig:app_itest_step}, with the slower system producing the reference (\textit{left}) and the quicker as the reference (\textit{right}). The non-linear and the linearised dynamics were evaluated.

\begin{figure}
    \centering
    \begin{subfigure}[t]{0.495\textwidth}
        \centering
        \includegraphics[width=\textwidth]{app_itest_slow}
        \caption{Slower system produces the reference output agent 1}
    \end{subfigure}
    \hfill
    \begin{subfigure}[t]{0.495\textwidth}
        \centering
        \includegraphics[width=\textwidth]{app_itest_fast}
        \caption{Quicker system produces the reference output of agent 1}
    \end{subfigure}
    \caption[Iterative Estimation -- Initial Input and Reference]{Response to the initial unit step input (training input and reference)}
    \label{fig:app_itest_step}
\end{figure}


\figref{fig:app_itest_lin_slow} shows the normalized transfer error of both methods \textit{tfest} and \textit{tfest+ILC} and how the inputs and corresponding outputs evolve over the first five iterations. In this figure, the slower \gls{ipc} system produced the reference output trajectory. A linearisation of the dynamics was used (so shown is a purely linear system).\\
\figref{fig:app_itest_nonlin_slow} then shows the same systems (with the slower one producing the reference output), but for the true non-linear dynamics.\\
In \figref{fig:app_itest_lin_fast} source and target are reversed, so quicker \gls{ipc} system produces the reference output trajectory. The systems are again approximated by the linearised \gls{ipc} systems. \\
In \figref{fig:app_itest_nonlin_fast}, the quicker dynamics are again used as the target reference output, but now the true non-linear system dynamics are simulated, which is the most difficult transfer case for those systems. This achieved the most fascinating results, as the iterative approach (\textit{red}) was able to recover from the poor (and potentially destabilised) initial transfer results which is seen in the approach without iterative uptdates to the transfer map (\textit{blue}). Using the iterative method, the peak input is smoothed down, which results in much better transfer (as it seemingly avoids the destabilizing inputs). However, since the iteratively estimated transfer matrix is still causal but cannot apply the peak input as initially aimed for, the output after the transfer now lags behind the desired output trajectory. This is seen in the initial phase after the step is applied and motivates the acausal lifted system matrix presented in \secref{sec:acausal_lsd_transfer}. While the new input trajectory seems as it would have been obtained using \gls{ilc} adaptations on the input, it was actually obtained using a transfer map from the initial step input. The transfer map can now be applied to other inputs as well, which makes this approach far superior to traditional \gls{ilc} directly on the inputs.

\begin{figure}
         \centering
         \includegraphics[width=0.95\textwidth]{app_itest_lin_slow.png}
         \caption[Iterative Estimation -- Linear Systems and Slow Reference]{Input transfer with and without iterative updates to the estimated input-transfer matrix for two linearised \acrshort{ipc} systems. The slower system is the reference system.}
         \label{fig:app_itest_lin_slow}
 \end{figure} 
 
 \begin{figure}
         \centering
         \includegraphics[width=0.95\textwidth]{app_itest_nonlin_slow}
         \caption[Iterative Estimation -- Non-Linear Systems and Slow Reference]{Input transfer with and without iterative updates to the estimated input-transfer matrix for two non-linear \acrshort{ipc} systems. The slower system is the reference system.}
         \label{fig:app_itest_nonlin_slow}
 \end{figure} 
 
  \begin{figure}
         \centering
         \includegraphics[width=0.95\textwidth]{app_itest_lin_fast}
         \caption[Iterative Estimation -- Linear Systems and Fast Reference]{Input transfer with and without iterative updates to the estimated input-transfer matrix for two linearised \acrshort{ipc} systems. The quicker system is the reference system.}
         \label{fig:app_itest_lin_fast}
 \end{figure} 
 
  \begin{figure}
         \centering
         \includegraphics[width=0.95\textwidth]{app_itest_nonlin_fast}
         \caption[Iterative Estimation -- Non-Linear Systems and Fast Reference]{Input transfer with and without iterative updates to the estimated input-transfer matrix for two non-linear \acrshort{ipc} systems. The quicker system is the reference system.}
         \label{fig:app_itest_nonlin_fast}
 \end{figure} 
 
 
It was tested, how the estimated input-transfer after the final iteration trial performs in the transfer of slightly different iputs. Therefore, the input-transfer was applied to step inputs with different gains (in contrast to the unit gain in the training) and to ramp inputs (a unit step input that takes "more time"). Results are shown for the slow target in \figref{fig:app_itest_gen_slow} and for the quick target in \figref{fig:app_itest_gen_fast}.

  \begin{figure}
         \centering
         \includegraphics[width=0.95\textwidth]{app_itest_gen_slow}
         \caption[Iterative Estimation -- Test Data and Fast Reference]{Input transfer with and without iterative updates to the estimated input-transfer matrix for different test inputs. The slower system is the reference system.}
         \label{fig:app_itest_gen_slow}
 \end{figure} 
 
  \begin{figure}
         \centering
         \includegraphics[width=0.95\textwidth]{app_itest_gen_fast}
         \caption[Iterative Estimation -- Test Data and Slow Reference]{Input transfer with and without iterative updates to the estimated input-transfer matrix for different test inputs. The quicker system is the reference system.}
         \label{fig:app_itest_gen_fast}
 \end{figure} 
 
 
 
\chapter{Parameters in Lifted System Description}
\graphicspath{{./Bilder/appendix/lifted_system_parameters}} 

\section{Lifted System Parameters}
This chapter shows the impact different parameters in the state space description of a stable second order, \gls{siso} \gls{lti} system on the parameters of their lifted system dynamics and their deviation system (in lifted system description).\\
The parameters of a lifted system correspond to the impulse response and can therefore be plotted over time. 
The following continuous system description is used
\begin{align}
	\vec{\dot x} &= \begin{bmatrix}
		0 & 1 \\
		-a_1 & -a_2
	\end{bmatrix} + 
	\begin{bmatrix}
		o \\ b_1
	\end{bmatrix} u\\
	y &= \begin{bmatrix}
		c_1 & 0
	\end{bmatrix} \vec{x}
\end{align}
The following steps were done:
\vspace{-0.5em}
\begin{itemize}[noitemsep, topsep=0pt]
    \item Transform the continuous dynamics into a discrete state-space model (for each parameter)
    \item Calculate the step responses seen in \figref{fig:app_stepres}.
    \item Use the discrete models to calculate the lifted dynamics in \figref{fig:app_lifted_params}
    \item Calculate input-transfer from a source system to the target system. The target system is seen in \figref{fig:app_lifted_params} as the most intense line in the plots with the last parameter value in the legend. The parameters of the transfer systems are seen in \figref{fig:app_lifted_transfer}.
    \item  Calculate the step responses using the transformation seen in \figref{fig:app_step_res_transfer}.
\end{itemize}

\begin{table}[h]
    \renewcommand{\arraystretch}{1.3}
    \begin{tabularx}{1\textwidth}{@{}lcccccX@{}}
        \toprule
        \textbf{Parameter} 	& $N$  	&  $T_s$ & $a_1$ & $a_2$ & $b_1$ & $c_1$ \\ \midrule
        \textbf{Value} 		& $100$ &  $0.02$ & $100$ & $5$ & $1$ & $1$ \\
        \bottomrule
    \end{tabularx}
    \caption{Parameters of the baseline system}
\end{table}

% STEP RESPONSE
\begin{figure}
    \centering
    \begin{subfigure}[t]{0.495\textwidth}
        \centering
        \includegraphics[width=\textwidth]{Step response of the lifted plant P for changes in a_1.pdf}
        \caption{Variation of $a_1$}
    \end{subfigure}
    \hfill
    \begin{subfigure}[t]{0.495\textwidth}
        \centering
        \includegraphics[width=\textwidth]{Step response of the lifted plant P for changes in a_2.pdf}
        \caption{Variation of $a_2$}
    \end{subfigure}
    \hfill
    \begin{subfigure}[t]{0.495\textwidth}
        \centering
        \includegraphics[width=\textwidth]{Step response of the lifted plant P for changes in b.pdf}
        \caption{Variation of $b_1$}
    \end{subfigure}
    \hfill
    \begin{subfigure}[t]{0.495\textwidth}
        \centering
        \includegraphics[width=\textwidth]{Step response of the lifted plant P for changes in c.pdf}
        \caption{Variation of $c_1$}
    \end{subfigure}
    \caption[Parameter Variations -- Impulse Response]{Impulse response for different parameter variations}
    \label{fig:app_stepres}
\end{figure}

% LIFTED SYSTEM PARAMETERS
\begin{figure}
    \centering
    \begin{subfigure}[t]{0.495\textwidth}
        \centering
        \includegraphics[width=\textwidth]{First column values of the lifted plant P for changes in a_1.pdf}
        \caption{Variation of $a_1$}
    \end{subfigure}
    \hfill
    \begin{subfigure}[t]{0.495\textwidth}
        \centering
        \includegraphics[width=\textwidth]{First column values of the lifted plant P for changes in a_2.pdf}
        \caption{Variation of $a_2$}
    \end{subfigure}
    
    \begin{subfigure}[t]{0.495\textwidth}
        \centering
        \includegraphics[width=\textwidth]{First column values of the lifted plant P for changes in b.pdf}
        \caption{Variation of $b_1$}
    \end{subfigure}
    \hfill
    \begin{subfigure}[t]{0.495\textwidth}
        \centering
        \includegraphics[width=\textwidth]{First column values of the lifted plant P for changes in c.pdf}
        \caption{Variation of $c_1$}
    \end{subfigure}
    \caption[Parameter Variations -- Lifted System Parameters]{Lifted system parameters for different parameter variations}
    \label{fig:app_lifted_params}
\end{figure}

% TRANSFER PARAMTERS
\begin{figure}
    \centering
    \begin{subfigure}[t]{0.495\textwidth}
        \centering
        \includegraphics[width=\textwidth]{First column values of the transformation matrix for different a_1.pdf}
        \caption{Variation of $a_1$}
    \end{subfigure}
    \hfill
    \begin{subfigure}[t]{0.495\textwidth}
        \centering
        \includegraphics[width=\textwidth]{First column values of the transformation matrix for different a_2.pdf}
        \caption{Variation of $a_2$}
    \end{subfigure}
    
    \begin{subfigure}[t]{0.495\textwidth}
        \centering
        \includegraphics[width=\textwidth]{First column values of the transformation matrix for different b.pdf}
        \caption{Variation of $b_1$}
    \end{subfigure}
    \hfill
    \begin{subfigure}[t]{0.495\textwidth}
        \centering
        \includegraphics[width=\textwidth]{First column values of the transformation matrix for different c.pdf}
        \caption{Variation of $c_1$}
    \end{subfigure}
    \caption[Parameter Variations -- Transfer Map Parameters]{Parameters of the transfer system for different parameter variations}
    \label{fig:app_lifted_transfer}
\end{figure}

% TRANSFER STEP RESPONSE
\begin{figure}
    \centering
    \begin{subfigure}[t]{0.495\textwidth}
        \centering
        \includegraphics[width=\textwidth]{Step response of the transformed input for different a_1.pdf}
        \caption{Variation of $a_1$}
    \end{subfigure}
    \hfill
    \begin{subfigure}[t]{0.495\textwidth}
        \centering
        \includegraphics[width=\textwidth]{Step response of the transformed input for different a_2.pdf}
        \caption{Variation of $a_2$}
    \end{subfigure}
    
    \begin{subfigure}[t]{0.495\textwidth}
        \centering
        \includegraphics[width=\textwidth]{Step response of the transformed input for different b.pdf}
        \caption{Variation of $b_1$}
    \end{subfigure}
    \hfill
    \begin{subfigure}[t]{0.495\textwidth}
        \centering
        \includegraphics[width=\textwidth]{Step response of the transformed input for different c.pdf}
        \caption{Variation of $c_1$}
    \end{subfigure}
    \caption[Parameter Variations -- Impulse Response with Output Transfer]{Impulse response including the deviation system (output-transfer) for different parameters variations}
    \label{fig:app_step_res_transfer}
\end{figure}


\section{SVD Dimensionality Reduction}\label{app:svd_dim_red}
The lifted system of the target system (baseline) are approximated using a linear combination of different numbers ($1-5$) of basis systems (principal components). Basis systems were identified with \gls{svd}. The results are seen in \figref{fig:app_approx_lifted}. The fit describes how well the dynamics of all systems can be described by those basis systems.\\
The same was done for the transformations. Shown is the approximation of the transformation from the first source system (first parameter with least intense line) to the target system using a linear combination of different numbers of basis transformations. The fit describes how well the transformations from all source systems to the target system can be described by the different number of basis transformations. The results are seen in \figref{fig:app_approx_transfer}.

\begin{figure}
    \centering
    \begin{subfigure}[t]{0.495\textwidth}
        \centering
        \includegraphics[width=\textwidth]{Approximate P for different a_1.pdf}
        \caption{Variation of $a_1$}
    \end{subfigure}
    \hfill
    \begin{subfigure}[t]{0.495\textwidth}
        \centering
        \includegraphics[width=\textwidth]{Approximate P for different a_2.pdf}
        \caption{Variation of $a_2$}
    \end{subfigure}
    
    \begin{subfigure}[t]{0.495\textwidth}
        \centering
        \includegraphics[width=\textwidth]{Approximate P for different b.pdf}
        \caption{Variation of $b_1$}
    \end{subfigure}
    \hfill
    \begin{subfigure}[t]{0.495\textwidth}
        \centering
        \includegraphics[width=\textwidth]{Approximate P for different c.pdf}
        \caption{Variation of $c_1$}
    \end{subfigure}
    \caption[SVD -- Lifted System Parameters]{Lifted system parameters of the target system approximated by a different amount of principal components}
    \label{fig:app_approx_lifted}
\end{figure}


\begin{figure}
    \centering
    \begin{subfigure}[t]{0.495\textwidth}
        \centering
        \includegraphics[width=\textwidth]{Approximate T_{12} for different a_1.pdf}
        \caption{Variation of $a_1$}
    \end{subfigure}
    \hfill
    \begin{subfigure}[t]{0.495\textwidth}
        \centering
        \includegraphics[width=\textwidth]{Approximate T_{12} for different a_2.pdf}
        \caption{Variation of $a_2$}
    \end{subfigure}
    
    \begin{subfigure}[t]{0.495\textwidth}
        \centering
        \includegraphics[width=\textwidth]{Approximate T_{12} for different b.pdf}
        \caption{Variation of $b_1$}
    \end{subfigure}
    \hfill
    \begin{subfigure}[t]{0.495\textwidth}
        \centering
        \includegraphics[width=\textwidth]{Approximate T_{12} for different c.pdf}
        \caption{Variation of $c_1$}
    \end{subfigure}
    \caption[SVD -- Transfer Map Parameters]{Parameters of the deviation system from the first source system (first parameter with least intense line color) to the target approximated by a different amount of principal components}
    \label{fig:app_approx_transfer}
\end{figure}




% ___________________________________________________________________
% ====================== COLLECTIVE ILC =======================
\chapter{Collective ILC with Trajectory Transfer}\label{ch:cilc}
\graphicspath{{./Bilder/appendix/cilc}} 
% ===================================================================
% -------------------------------------------------------------------

Extending the CILC framework by an input transfer is visualized in \figref{fig:app_cilc_with_input_transfer}.

In some systems, input-transfer might not be easily estimated. In \figref{fig:app_cilc_with_output_transfer} a method is illustrated that makes use of the output transfer to find agent specific best performing input trajectories. Instead of updating the input before the ILC update as in the input transfer case, an error trajectory is estimated that corresponds to the best input for each agents. 
\begin{figure}
        \centering
        \includegraphics[width=0.6\textwidth]{cilc_with_input_transfer}
        \caption[CILC with Input Transfer]{CILC with input transfer}
        \label{fig:app_cilc_with_input_transfer}
\end{figure}

\begin{figure}
        \centering
        \includegraphics[width=1\textwidth]{cilc_with_output_transfer}
        \caption[CILC with Input Transfer]{CILC with output transfer}
        \label{fig:app_cilc_with_output_transfer}
\end{figure}


%\chapter{Playground for Layout and Latex Commands}
\graphicspath{{./Bilder/transfer_methods}} 

\begin{example}\label{example:test}
	Test ev
	\begin{equation}
		3 = 3 \label{eq:testi}
	\end{equation}
	This is test text with \eqref{eq:testi}.

%	\begin{minipage}[c]{1\linewidth}	
	\begingroup
		\centering
	    \captionsetup{type=figure, labelfont={color=example_font_color, bf}, font={color=example_font_color}}		
		\tabskip=0pt
		\valign{#\cr
		  \hbox{%
		    \begin{subfigure}[b]{.495\textwidth}
		    \centering
		        \includegraphics[width=0.95\textwidth]{/perfect transfer bode.pdf}
		        \caption{Bode Diagram}
		        \label{subfig:testi}
		    \end{subfigure}%
		  }\cr
		  \noalign{\hfill}
		  \hbox{%
		    \begin{subfigure}{.495\textwidth}
		    \centering
		        \includegraphics[width=0.95\textwidth]{/perfect transfer slow time.pdf}
		        \caption{Sine Input slow}
		    \end{subfigure}%
		  }\vfill
		  \hbox{%
		    \begin{subfigure}{.495\textwidth}
		    \centering
		        \includegraphics[width=0.95\textwidth]{/perfect transfer fast time.pdf}
		        \caption{Sine Input fast}
		    \end{subfigure}%
		  }\cr
		}
	\caption{Input transfer for systems ... and ... using the ideal input transfer}
	\label{fig:testi}
	\endgroup

%	\end{minipage}
	See u in \figref{subfig:testi} and \figref{fig:testi}. \\
	\textbf{Conclusions}
	\vspace{-0.5em}
	\begin{itemize}[noitemsep, topsep=0pt]
		\item When choosing an input transfer approach, it is important to define the requirements for the transfer with regards to the expected input frequencies, complexity of the transfer and largest allowed transfer error.
		\item Using the minimum $\|\cdot\|_{\infty}$-norm of the error dynamics to evaluate transfer can be disadvantageous, as a low $\|\cdot\|_{\infty}$-norm might come at the cost of very poor transfer performance over a certain range of input frequencies, which could have been much better otherwise.	
		\item A constant gain has limited abilities to decrease the transfer error, as the ideal constant gain that minimizes the output error depends the input frequency. (I.e a gain can be chosen to minimize the error for slow frequencies or to minimize the overall maximum error).  The benefit is very limited to small frequency ranges.
		\item A constant gain cannot account for difference in phase. It is impossible, to achieve perfect output alignment for an arbitrary frequency using a constant gain. 
		\item Adding a time-shift to a constant gain can perfectly align two outputs at an arbitrary frequency. The benefit is limited to this frequency only, as it cannot compensate for frequency dependence 
		\item The error can be removed completely for all frequencies with the ideal input-transfer
		\item A reduced order model of the ideal input-transfer can achieve good transfer performance over a long range of input frequencies. 
	\end{itemize}
\end{example}

\section{Glossar Entries Playground}
Acronyms:\\
Gls (first use): \highlight{\Gls{lttm}}\\
Second use: \gls*{lttm}\\
Acrolongpl: \acrlongpl{lttm}\\
Acroshort: \acrshort{lttm}\\
\gls{lti}\gls{lqr}\gls{siso}\gls{ipc}\gls{twipr}\gls{pca}\gls{svd}\gls{nrmse}\gls{rmse}\gls{rms}\gls{mas}\gls{prbs}\gls{bibo}\gls{todo}\\
Glossar:\\
\gls{dcgain}\gls{transfer_map}\gls{deviation_system}\gls{input_transfer}\gls{output_transfer}\gls{error_dynamics}\\
\Gls{transfer_error_dynamics}\\
\gls{sim2real}\\

\lipsum[1] %\hyperlink{gls:transfer_error_dynamics}{\gls*{transfer_error_dynamics}}
\glsc{transfer_error_dynamics}




