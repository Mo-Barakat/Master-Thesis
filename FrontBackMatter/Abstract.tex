
\selectlanguage{english}
\begin{abstract}
%General Problem
The robotic industry is growing rapidly.
Robots are becoming more and more capable and are used in a wide range of applications.
The design, modeling and control of robotic systems is a research field that is constantly evolving.

%% Research Gap
However, there are still many challenges to overcome.
The design of robotic systems is still a complex and time-consuming process, particularly in regard to multi-legged robots that should be able to execute complex maneuvers.
The control of robotic systems is still a challenging task, especially when it comes to dynamic and complex systems.
The search for the optimal control strategy is still an open research question.

%% Content
This thesis presents an in-depth exploration of the dynamic control and design of robotic systems.
The main focus is on the design and control of a multi-legged robot.
The robot is designed and built from scratch, including the mechanical and electronic design.
A combination of theoretical modeling and practical implementation were used to achieve the desired objectives.
The research first focuses on the design of a multi-legged robot that is capable of executing complex maneuvers. The robot is then modeled and simulated in a virtual environment.
The simulation is used to test and evaluate the robot's performance.
The robot is then built and tested in real life.
The research investigates the main elements that affect maneuverability and stability, such as the integration of multiple degrees of freedom, structural integrity, and weight distribution.


%% Contributions
Significant results are achieved in the design and control of the robot.
The robot demonstrates the capability to maintain balance while adjusting its hip and knee joint angles, as well as exhibits stable locomotion on its two wheels, effectively navigating within the simulated environment.
The robot's design is optimized to find the best trade-off between stability and maneuverability.
Different problems are addressed throughout the research, such as the design complexity to meet the mobility requirements, Incorporating the changing location of the center of mass in the dynamic model, and retuning the control of the robot when one or both of the joint angles change.
The developed Legged two-wheeled inverted pendulum robot serves as a platform for future research to build upon and test different control strategies.

%Furthermore
This research contributes valuable insights into the field of robotics, offering practical solutions for the challenges faced in designing and controlling wheeled and bipedal robots.
It lays a foundation for future innovations, aiming to broaden the scope and capabilities of robotic systems in real-world applications.




%Overall, this thesis provides a valuable contribution to the field of
%%%
\par
\keywords{Robotics, Multi-Legged Robot, Control, Simulation, Design, CAD, Modeling, Inverted Pendulum, Two-Wheeled bi-pedal legged robot}
\end{abstract}



