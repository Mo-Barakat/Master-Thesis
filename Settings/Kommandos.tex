% Insert custom commands here. Some predefined commands can be found below.

% Creates the diag operator 
\DeclareMathOperator{\diag}{diag}
% Creates a partial derivative (a \partial above and below)
\newcommand{\pfrac}[2]{\frac{\partial #1}{\partial #2}}
% "Total" derivative
\newcommand{\tdfrac}[2]{\frac{\mathrm{d} #1}{\mathrm{d} #2}}
% Differentials at the end of an integral
\renewcommand{\d}{\,\mathrm{d}}
% "Partial" symbol
\newcommand{\p}{\partial}
% Bold symbols (including Greek)
\newcommand{\bs}[1]{\boldsymbol{#1}}
% Absolute value
\newcommand{\abs}[1]{\left\lvert #1 \right\rvert}
% Norm
\newcommand{\norm}[1]{\left\lVert #1 \right\rVert}
% Divergence
\DeclareMathOperator{\divv}{div}
% Gradient
\DeclareMathOperator{\grad}{grad}
% Curl
\DeclareMathOperator{\rot}{rot}
% Cofactor
\DeclareMathOperator{\cof}{Cof}
% Trace
\DeclareMathOperator{\Spur}{Sp}
% Jump bracket
\newcommand{\jump}[1]{\left \llbracket #1 \right \rrbracket}
% Physical components
\newcommand{\phy}[1]{{\langle #1 \rangle}}
% Inverse of a symbol, with adjusted virtual height,
% so that indices fit (e.g., \inverse{\sigma}^{ij} )
\newcommand{\inverse}[1]{\stackrel{-1}{#1}\!\!\vphantom{#1}}
% Fixes broken indentations, e.g., after aligns
\newcommand{\fixIndent}{ \\[-\parskip]}
% Abbreviations for creating equations
\newcommand{\beq}{\begin{equation}}
	\newcommand{\eeq}{\end{equation}}
\newcommand{\beqq}{\begin{equation*}}
	\newcommand{\eeqq}{\end{equation*}}
% Formula symbols with subscripted formula (or character), with index height adjustment.
\newcommand{\sU}[2]{\underset{#2}{#1}\vphantom{#1}}
% Like above, but ignores the width of the formula in the typesetting
\newcommand{\sUs}[2]{\smash{\underset{#2}{#1}\vphantom{#1}}}

%% Neue Referenzbefehle über varioref (kein fancyref, da dies keine Abkürzungen vorsieht)
%\newcommand{\charef}[1]{Kap.\,\vref*{#1}}
%\newcommand{\secref}[1]{Abschn.\,\vref*{#1}}
%\newcommand{\ssecref}[1]{Unterabschn.\,\vref*{#1}}
%\newcommand{\veqref}[1]{Gl.\,\eqref{#1}\vpageref{#1}}
%\newcommand{\figref}[1]{Abb.\,\vref*{#1}}
%\newcommand{\sfigref}[2]{Abb.\,\ref{#1}\subref{#2}\vpageref{#1}}
%\newcommand{\tabref}[1]{Tab.\,\vref*{#1}}
%\newcommand{\lstref}[1]{Lst.\,\vref*{#1}}
%% Ohne Seitenangabe
%\newcommand{\neqref}[1]{Gl.\,\eqref{#1}}
%\newcommand{\ncharef}[1]{Kap.\,\ref*{#1}}
%\newcommand{\nsecref}[1]{Abschn.\,\ref*{#1}}
%\newcommand{\nssecref}[1]{Unterabschn.\,\ref*{#1}}
%\newcommand{\nfigref}[1]{Abb.\,\ref*{#1}}
%\newcommand{\nsfigref}[2]{Abb.\,\ref{#1}\subref{#2}}
%\newcommand{\ntabref}[1]{Tab.\,\ref*{#1}}
%\newcommand{\nlstref}[1]{Lst.\,\ref*{#1}}

%%  ENGLISH VERSION
%% Neue Referenzbefehle über varioref (kein fancyref, da dies keine Abkürzungen vorsieht)
%\newcommand{\charef}[1]{ch.\,\vref*{#1}}
%\newcommand{\secref}[1]{sec.\,\vref*{#1}}
%\newcommand{\ssecref}[1]{subsec.\,\vref*{#1}}
%\newcommand{\veqref}[1]{eq.\,\eqref{#1}\vpageref{#1}}
%\newcommand{\figref}[1]{Fig.\,\vref*{#1}}
%\newcommand{\sfigref}[2]{fig.\,\ref{#1}\subref{#2}\vpageref{#1}}
%\newcommand{\tabref}[1]{tab.\,\vref*{#1}}
%\newcommand{\lstref}[1]{lst.\,\vref*{#1}}
%% Ohne Seitenangabe
%\newcommand{\neqref}[1]{eq.\,\eqref{#1}}
%\newcommand{\ncharef}[1]{ch.\,\ref*{#1}}
%\newcommand{\nsecref}[1]{sec.\,\ref*{#1}}
%\newcommand{\nssecref}[1]{subsec.\,\ref*{#1}}
%\newcommand{\nfigref}[1]{fig.\,\ref*{#1}}
%\newcommand{\nsfigref}[2]{fig.\,\ref{#1}\subref{#2}}
%\newcommand{\ntabref}[1]{tab.\,\ref*{#1}}
%\newcommand{\nlstref}[1]{lst.\,\ref*{#1}}
%% Anwendung z. B. mit: ...siehe \charef{cha:01}  -> ... siehe Kap. 3 auf S. 2
%% Unterabbildungen z. B. mit: ... siehe \sfigref{fig:01}{sfig:01:a} -> ... siehe Abb. 3.2(a)

%%%%%%%%%%%%%%%%%%%%%%%%%%%%%%%%%%%%%%%%%%%%%%%%%%%%%%%%%%%%%%%%%%%%%%%%%
% Use this command for a filename
\newcommand{\filenamecode}[1]{\texttt{#1}}

%Appendix Reference
\newcommand*{\appref}[1]{%
	\ifcase\colorfulLayout
	\or
	  \begingroup
	    \hypersetup{linkcolor=marker_blue}%
	    Appendix\,\ref{#1}%
	  \endgroup
	\or
	  \begingroup
	    \hypersetup{linkcolor=black}%
	    Appendix\,\ref{#1}%
	  \endgroup
	\fi
}


%Section Reference
\newcommand*{\secref}[1]{%
	\ifcase\colorfulLayout
	\or
	  \begingroup
	    \hypersetup{linkcolor=marker_blue}%
	    Sec.\,\ref{#1}%
	  \endgroup
	\or
	  \begingroup
	    \hypersetup{linkcolor=black}%
	    Sec.\,\ref{#1}%
	  \endgroup
	\fi
}

%Chapter Reference
\newcommand*{\charef}[1]{%
	\ifcase\colorfulLayout
	\or
	  \begingroup
	    \hypersetup{linkcolor=marker_blue}%
	    Ch.\,\ref{#1}%
	  \endgroup
	\or
	  \begingroup
	    \hypersetup{linkcolor=black}%
	    Ch.\,\ref{#1}%
	  \endgroup
	\fi
}

%\newcommand{\noteref}[1]{note on p.\pageref{#1}}
\newcommand*{\noteref}[1]{%
	\ifcase\colorfulLayout
	\or
	  \begingroup
%	    \hypersetup{linkcolor=example_color}%
	    Note~\ref{#1}%
	  \endgroup
	\or
	  \begingroup
	    \hypersetup{linkcolor=black}%
	    Note~\ref{#1}%
	  \endgroup
	\fi
}


% Item Reference
\newcommand{\itemref}[1]{\begingroup\hypersetup{linkcolor=marker_blue}\ref{#1}\endgroup}


%Equation Reference
\renewcommand*{\eqref}[1]{%
	\ifcase\colorfulLayout
	\or
	  \begingroup
	    \hypersetup{linkcolor=scioi_blue}%
	    (\ref{#1})%
	  \endgroup
	\or
	  \begingroup
	    \hypersetup{linkcolor=black}%
	    (\ref{#1})%
	  \endgroup
	\fi
}

% Figure Reference
\newcommand*{\figref}[1]{%
	\ifcase\colorfulLayout
	\or
	  \begingroup
	    \hypersetup{linkcolor=marker_blue}%
	    Fig.\,\ref{#1}%
	  \endgroup
	\or
	  \begingroup
	    \hypersetup{linkcolor=black}%
	    Fig.\,\ref{#1}%
	  \endgroup
	\fi
}

% Table Reference
\newcommand*{\tabref}[1]{%
	\ifcase\colorfulLayout
	\or
	  \begingroup
	    \hypersetup{linkcolor=marker_blue}%
	    Tab.\,\ref{#1}%
	  \endgroup
	\or
	  \begingroup
	    \hypersetup{linkcolor=black}%
	    Tab.\,\ref{#1}%
	  \endgroup
	\fi
}

% Example Reference
\newcommand*{\exref}[1]{%
	\ifcase\colorfulLayout
	\or
	  \begingroup
	    \hypersetup{linkcolor=example_color}%
	    Ex.\,\ref{#1}%
	  \endgroup
	\or
	  \begingroup
	    \hypersetup{linkcolor=black}%
	    Ex.\,\ref{#1}%
	  \endgroup
	\fi
}


% Text Highlight
%\newcommand{\highlight}[1]{
%	\ifcase\colorfulLayout 
%	\or\textcolor{marker_blue}{\textbf{#1}}\or\textit{#1}\fi
%	}
\newcommand{\highlight}[1]{\textcolor{marker_blue}{\textbf{#1}}}
	
% Test
%\let\oldacrshort\acrshort
%\renewcommand{\acrshort}[1]{\textbf{\oldacrshort{#1}}}

% General
%\renewcommand{\vec}[2][]{\ensuremath{\prescript{}{\mathcal{#1}}{\mathbf{#2}}}}
\renewcommand{\vec}[1]{\mathbf{#1}}
\renewcommand{\norm}[1]{\ensuremath{\left\lVert#1\right\rVert}}
% Transfer error
\newcommand{\err}{\ensuremath{\varepsilon_{\vec{u}}^{(i\rightarrow j)}}}
\newcommand{\errn}{\ensuremath{\varepsilon_{\vec{u},\mathrm{NRMS}}^{(i\rightarrow j)}}}

\newcommand{\inv}{\mathrm{inv}}
% ENVIRONMENTS

\newenvironment{todobox}{
\hypersetup{linkcolor=red!75!black} %linkcolor=red!75!black
\begin{tcolorbox}[boxsep=1pt, coltext=red!75!black, colback=red!5!white,colframe=red!75!black, boxrule=1pt,  left = 1pt, top = 1pt, bottom=1pt, right = 1pt, title=\begingroup \hypersetup{linkcolor=white}  \endgroup]  %title=\textbf{ToDo}
%\acrshort{todo} 
}{\end{tcolorbox}}

% NOTE BOX
\newcounter{notecount}
\counterwithin{notecount}{chapter}
% env
\newenvironment{notebox}[1][]{
%\let\oldacrshort\acrshort
%\renewcommand{\acrshort}[1]{\hypersetup{linkcolor=note_color}{\oldacrshort{##1}}}
\refstepcounter{notecount}
\hypersetup{linkcolor=note_color}
\begin{tcolorbox}[boxsep=1pt, coltext=note_color, colback=note_color!5!white,colframe=note_color!90!black, boxrule=1pt,  left = 1pt, top = 1pt, bottom=1pt, right = 1pt, title=\ifthenelse{\equal {#1}{} }
	{\textbf{Note~\thenotecount} } % if no example titel given, print only "Example 5.12" with number
	{\textbf{Note~\thenotecount : #1} }] % if titel given, add colon "Example 5.12: Titel"
}{\end{tcolorbox}}
%title=\textbf{Note~\thenotecount #1}


\newenvironment{question}{\begin{tcolorbox}[boxsep=1pt, coltext=question_color, colback=question_color!5!white,colframe=question_color!75!black, boxrule=1pt,  left = 1pt, top = 1pt, bottom=1pt, right = 1pt, title=\textbf{Question}]
%\setbeamercolor{itemize/enumerate body}{fg=question_color}
%\setbeamercolor{item}{fg=question_color}
}{\end{tcolorbox}}

%agent_2
%ressources_color
\newenvironment{ressources}{
%\hypersetup{linkcolor=pink_link}
\begin{tcolorbox}[boxsep=1pt, coltext=ressources_color!80!black, colback=ressources_color!15!white,colframe=ressources_color!85!black, boxrule=1pt,  left = 1pt, top = 1pt, bottom=1pt, right = 1pt, title=\textbf{Ressources}]
%\setbeamercolor{itemize/enumerate body}{fg=ressources_color!85!black}
%\setbeamercolor{item}{fg=ressources_color!85!black}
}{\end{tcolorbox}}

\newenvironment{outlook}{
%\let\oldacrshort\acrshort
%\renewcommand{\acrshort}[1]{\hypersetup{linkcolor=note_color}{\oldacrshort{##1}}}
\hypersetup{linkcolor=note_color}
\begin{tcolorbox}[boxsep=1pt, coltext=note_color, colback=note_color!5!white,colframe=note_color!90!black, boxrule=1pt,  left = 1pt, top = 1pt, bottom=1pt, right = 1pt, title=\textbf{Outlook}]
%\setbeamercolor{itemize/enumerate body}{fg=note_color}
%\setbeamercolor{item}{fg=note_color}
}{\end{tcolorbox}}

% EXAMPLE BOX
\newcounter{examplecount}
\counterwithin{examplecount}{chapter}
% env
\newenvironment{example}[1][]{
%\hypersetup{linkcolor=note_color}
\refstepcounter{examplecount}
\begin{tcolorbox}[breakable, boxsep=1pt, coltext=example_font_color, colback=example_color!5!white, colframe=example_color, boxrule=1pt,  left = 1pt, top = 1pt, bottom=1pt, right = 1pt, 
title=\ifthenelse{\equal {#1}{} }
	{\textbf{Example~\theexamplecount} } % if no example titel given, print only "Example 5.12" with number
	{\textbf{Example~\theexamplecount : #1} }] % if titel given, add colon "Example 5.12: Titel"
}{\end{tcolorbox}}
% , float=t
% example_color!3!white
%title=\textbf{Example~\theexamplecount #1}

% Simple Red Todobox (Inline)
\newtcbox{\todo}[1][red]{on line,
arc=3pt,colback=#1!5!white,colframe=#1!75!black,
before upper={\rule[-3pt]{0pt}{10pt}},boxrule=1pt,coltext=#1,
boxsep=0pt,left=2pt,right=2pt,top=1pt,bottom=.5pt}


% custom gls command that links to Glossary, but is not listed (has link color)
\newcommand{\glsc}[1]{\hyperlink{gls:#1}{\gls*{#1}}}
\newcommand{\Glsc}[1]{\hyperlink{gls:#1}{\Gls*{#1}}}
\newcommand{\glsplc}[1]{\hyperlink{gls:#1}{\glspl*{#1}}}
\newcommand{\Glsplc}[1]{\hyperlink{gls:#1}{\Glspl*{#1}}}
\newcommand{\glslinkc}[2]{\hyperlink{gls:#1}{\glslink*{#1}{#2}}}


%\newcommand{\glsc}[1]{\hyperlink{gls:#1}{\gls*{#1}}}
%\newcommand{\Glsc}[1]{\hyperlink{gls:#1}{\Gls*{#1}}}
%\newcommand{\glsplc}[1]{\hyperlink{gls:#1}{\glspl*{#1}}}
%\newcommand{\Glsplc}[1]{\hyperlink{gls:#1}{\Glspl*{#1}}}
%\newcommand{\glslinkc}[2]{\hyperlink{gls:#1}{\glslink*{#1}{#2}}}

% Avoid page break (does not work)
%\newcommand\mynobreakpar{\par\nobreak\@afterheading} 

% Keywords command
\providecommand{\keywords}[1]
{
  \small	
  \textbf{\textit{Keywords---}} #1
}

