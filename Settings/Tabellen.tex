% Tabellen mit fester Gesamtbreite
\usepackage{tabularx}
% Beispiel: \begin{tabularx}{6cm}{lrX} <- X ist die variable Größe

% Mehrseitige, lange Tabellen
\usepackage{longtable}
% Wichtig: Setzen von \endfirsthead, \endhead, \endfoot, \endlastfoot am Anfang nach \begin{longtable}

% Ermöglichung von gedrehten Tabellen
\usepackage{rotating}
% gedrehte Tabelle mit der neuen Float-Tabellenumgebung \begin{sidewaystable} erstellen

% Professionelles Tabellenlayout
\usepackage{booktabs}
% Ermöglicht die Verwendung von \toprule, \midrule, \bottomrule, \cmidrule
% Wichtig für wissenschaftlich und ästhetisch ansprechende Tabellen
% add \ra{1.3} between \begin{table} and \begin{tabularx} to increase size between rows
\newcommand{\ra}[1]{\renewcommand{\arraystretch}{#1}}

\usepackage{ltablex} % for tables that span multiple pages in combination with tabularx
\usepackage{multirow} % form multirow cells in tables