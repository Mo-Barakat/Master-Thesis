% Custom colors
\usepackage{color}

\definecolor{scioi_blue}{RGB}{103, 206, 236} % light blue
\definecolor{marker_blue}{RGB}{31, 119, 179}
\definecolor{question_color}{RGB}{227, 127, 45}
\definecolor{note_color}{RGB}{31, 119, 179}
\definecolor{ressources_color}{RGB}{207, 177, 205}
\definecolor{example_color}{RGB}{90, 90, 90} %grey
\definecolor{example_font_color}{RGB}{60, 60, 60} % dark grey

\definecolor{agent_1}{RGB}{61,83,149} %dark blue
\definecolor{agent_2}{RGB}{223,96,125} %pink
\definecolor{pink_link}{RGB}{236, 0, 141} %pink for links





% Ermöglicht die Erweiterung von pdf-Dateien mit Links und Ähnlichem.
\usepackage{hyperref}
% Wenn Linkstellen im Text farbig sein sollen oder gar Formeln gesetzt werden, \texorpdfstring nutzen
% Mit \hypersetup werden die Schalter dieses Pakets weiter modifiziert


% Wichtige Macro-Definitionen  % vorher bei Hilfpaketen
\usepackage{etoolbox}
% Z. B. wichtig für logische Abfragen:
% \ifstrequal{String}{Vergleichstring}{...passiert when gleich}{...ansonsten}
% Anderes Beispiel, s. "Anpassen der Tiefe von nummerierten Überschriften" in Satzeinstellungen


% Dokumenttitel setzen
\hypersetup{pdftitle={\Titel}}

% Autor setzen
\hypersetup{pdfauthor={\Autor}}

% Thema setzen
\hypersetup{pdfsubject={\Untertitel}}

% Einstellung der Linkfarben (Drucken oder reine PDF-Ansicht)
\ifcase\pdftype 
\or % Für Drucken
\hypersetup{
%    colorlinks=true,
%    citecolor=black,
%    filecolor=black,
%    linkcolor=black,
%    menucolor=black,
%    urlcolor=black
   colorlinks=true,
   linkcolor=marker_blue, %blue,
	anchorcolor=black,
   citecolor=marker_blue, %blue,
   filecolor=magenta,
   menucolor=blue,
   urlcolor=pink_link
}
\or % Für reine PDF-Verwendung
\hypersetup{
   colorlinks=true,
   linkcolor=marker_blue, %blue,
	anchorcolor=black,
   citecolor=marker_blue, %blue,
   filecolor=magenta,
   menucolor=blue,
   urlcolor=blue
}
\fi

% Einstellung, was das Inhaltsverzeichnis als Link verwenden soll
\ifcase\pdftype 
\or % Für Drucken
\hypersetup{linktoc=all}
\or % Für reine PDF-Verwendung
\hypersetup{linktoc=page}
\fi

% Einstellen der Bookmarks
\hypersetup{
bookmarksopen=true,
bookmarksnumbered=true
}

% Setzt Links korrekt für Floats (Fixt, das Labels nicht mehr am Ende stehen müssen)
\usepackage[all]{hypcap}

% Schönere URL-Formatierung (Für BibTeX und Text) von http://www.kronto.org/thesis/tips/url-formatting.html
\makeatletter
\def\url@leostyle{\@ifundefined{selectfont}{\def\UrlFont{\sf}}{\def\UrlFont{\small\ttfamily}}}
\makeatother
\urlstyle{leo}
% Manuelles Hinzufügen einer URL mit \url{www.narf.com}

