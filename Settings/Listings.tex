% Änderungen, um Listings in KOMA korekt darzustellen (repariert die Floats)
\usepackage{scrhack}

% Listings-Unterstützung
\usepackage{listings}

% Bennenung der Caption und des Verzeichnisses
\renewcommand*\lstlistingname{Lst.\!}
\renewcommand*\lstlistlistingname{Listings}

\lstset{
breaklines=true,		% Automatischer Umbruch
basicstyle=\scriptsize		% Etwas kleinerer Text als in der Umgebung
}

% Ein Beispiel:
%\begin{lstlisting}[float, language=C, label=lst:01, caption={Ein C-Beispiel}, frame=trBL]
%int main()
%{
%	int i;
%	for (i = 1; i < 10; i++)
%		printf("%i",i);
%	return(0);
%}
%\end{lstlisting}
\usepackage{listings}
\usepackage{color}
\definecolor{codegreen}{rgb}{0,0.6,0}
\definecolor{codegray}{rgb}{0.5,0.5,0.5}
\definecolor{codepurple}{rgb}{0.58,0,0.82}
\definecolor{backcolour}{rgb}{0.95,0.95,0.92}

\lstdefinestyle{mystyle}{
	backgroundcolor=\color{backcolour},   
	commentstyle=\color{codegreen},
	keywordstyle=\color{magenta},
	numberstyle=\tiny\color{codegray},
	stringstyle=\color{codepurple},
	basicstyle=\ttfamily\footnotesize,
	breakatwhitespace=false,         
	breaklines=true,                 
	captionpos=b,                    
	keepspaces=true,                 
	numbers=left,                    
	numbersep=5pt,                  
	showspaces=false,                
	showstringspaces=false,
	showtabs=false,                  
	tabsize=2
}

\lstset{style=mystyle}
