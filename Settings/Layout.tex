% Hier sind Layoutoptionen zu finden, die aus Übersichtsgründen nicht in der Datei "Klasse.tex" eingefügt sind

% Serifenschrift 
% Ersetzt Computer Modern mit Latin Modern, sieht aber nahezu gleich aus. Grund ist die mangelnde Unterstützung von korrekt gesetzten Umlauten und die Lesbarkeit am Monitor.
\usepackage{lmodern}

% Fügt Symbole zum Font hinzu, die in Latin Modern nicht enthalten sind, wie z. B. das Euro-Symbol
\usepackage{textcomp}

% Ersetzt die Schreibmaschinen-Schrift und verwendet Adobe Courier statt Latin Modern oder Computer Modern
\usepackage{courier}

% Optischer Randausgleich
\usepackage{microtype}
% Wenn man die Augen zukneift, kann man ohne diesem Paket bei Blocksatz auf der rechten Seite "Dellen" erkennen, da Punkte und Striche vor der virtuellen Trennlinie gehalten werden. Dieses Paket ermöglicht kleine Übertretungen, die das optische Bild verbessern.

% Setzt die Schriftgröße. Ebenfalls soll die Titelseite separat gesetzt werden
\KOMAoptions{
fontsize=10pt,			% Schriftgröße. Standard = 11pt
titlepage=true			% Separat gesetzte Titelseite verwenden
}

% Verwendung des geometry Packets für die Einstellung des Papierformats / Seitenabstände / Bindekorrektur
\usepackage[
a4paper, 													% DIN A4-Papier
  inner=25mm,
  outer=25mm,
  top=25mm,
  bottom=25mm,
  heightrounded,
  marginparwidth=51pt,
  marginparsep=17pt,
  headsep=24pt,
%\ifcase\pagetype \or twoside=false \or twoside=true \fi , 	% Ein- oder zweiseitig 
bindingoffset=\bindCorrection								% Binde-Korrektur
]{geometry}

% Header und Footer für ein- oder zweiseitigen Satz über das Paket scrpage2 einstellen
\usepackage[automark, headsepline, footsepline, plainfootsepline, plainheadsepline, ilines]{scrlayer-scrpage}
\pagestyle{scrheadings}
%\clearscrheadfoot repaced by \clearpairofpagestyles 
\clearpairofpagestyles
\ifcase\pagetype
\or % Für einseitigen Satz
\lohead[\HeaderTitel]{\HeaderTitel}
\rohead[\pagemark]{\pagemark}
\lofoot[{\scriptsize \leftmark}]{
\ifthenelse{\equal{\leftmark}{\rightmark}}
	{\scriptsize \rightmark}
{
	\ifthenelse{\equal{\rightmark}{}}
		{\scriptsize {\leftmark}}
		{\scriptsize {\leftmark} -- {\rightmark}}
}
}
\or % Für zweiseitigen Satz
\ohead[\pagemark]{\pagemark}
\lohead[\HeaderTitel]{\HeaderTitel}
\refoot[{\scriptsize {\leftmark}}]{{\scriptsize {\leftmark}}}
\lofoot[{\scriptsize {\rightmark}}]{{\scriptsize {\rightmark}}}
\fi

% New switch to disable Chapter 0. for Notation in footer
\newif\ifchanum
\chanumtrue
% Je nach gewählter Klasse das left- und rightmark anpassen
\ifcase\doctype 
\or % Für scrartcl
\addtokomafont{section}{\thispagestyle{scrplain}}
\renewcommand{\sectionmark}[1]{\markright{}\markleft{Abschnitt\ \thesection.\ #1}{}}
\renewcommand{\subsectionmark}[1]{\markright{#1}{}}
\or % Für scrreprt 
\renewcommand{\chaptermark}[1]{\markright{}\markleft{\ifchanum \chaptername\ \thechapter.\ #1 \else #1\fi}{}}
%\renewcommand{\chaptermark}[1]{\markright{}\markleft{\chaptername\ \thechapter.\ #1}{}} %original
\renewcommand{\sectionmark}[1]{\markright{#1}{}}
\or % Für scrbook
\renewcommand{\chaptermark}[1]{\markright{}\markleft{\chaptername\ \thechapter.\ #1}{}}
\renewcommand{\sectionmark}[1]{\markright{Abschnitt\ \thesection.\ #1}{}}
\fi


