% Index Erstellung
\usepackage{makeidx}

% Verweise auf andere Index-Einträge als deutsche Abkürzung
\renewcommand{\seename}{s.}

% Index festsetzen
\makeindex

% Achtung: makeindex muss folgendermaßen aufgerufen werden:
% makeindex -s "Index/Index.ist -i Dokument.idx -o Dokument.ind
%  Argument z. B. bei TexNicCenter im Postprozessor als
% -s "Index/Index.ist -i "%bm.idx" -o "%bm.ind"

% Beispiele für die Anwendung
% \index{hello} 							hello, 1 						Einfacher Eintrag
% \index{hello!Peter} 	  		Peter, 3 						Untereintrag von 'hello'
% \index{Sam@\textsl{Sam}} 		Sam, 2 							Formatierter Eintrag (Sortierung vorm @)
% \index{Lin@\textbf{Lin}} 		Lin, 7 							Wie zuvor
% \index{Jenny|textbf} 				Jenny, 3 						Formateirte Seitenzahl
% \index{Joe|textit} 					Joe, 5 							Wie oben
% \index{ecole@\'ecole} 			école, 4 						Sonderzeichen behandeln
% \index{Peter|see{hello}} 		Peter, s. hello 		Kreuzreferenz

