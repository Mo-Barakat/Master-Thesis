% Paket für schnellen Beispieltext zum Füllen
\usepackage{blindtext}
% Aus der Hilfe:
% \blindtext creates some text,
% \Blindtext creates more text.
% \blinddocument creates a small document with sections, lists...
% \Blinddocument creates a large document with sections, lists...

% etools

% Weitertes Macro-Paket. Zwar depreciated, dennoch benötigt, da es in Skripten Kommandos auflöst
\usepackage{ifthen}
% Bsp: \ifthenelse{\equal{\kommando1}{\kommando2}}{true}{false}

% Ermöglicht das Verwenden von Listen mit kleineren vertikalen Abständen
%\usepackage{mdwlist}
%% Statt itemize einfach itemize* verwenden. Analog dazu existiert enumerate* und description*
%% Diese Umbegungen erlauben auch Unterbrechungen, z. B.  mit \suspend{itemize*} bla bla \resume{itemize*}
% stattdessen \begin{itemize}[noitemsep,topsep=0pt], bzw \begin{enumerate}[resume*]
\usepackage[inline]{enumitem}

% for todolists
\newlist{todolist}{itemize}{2}
\setlist[todolist]{label=$\square$,noitemsep, topsep=0pt}
\usepackage{pifont}
\newcommand{\cmark}{\ding{51}}%
\newcommand{\xmark}{\ding{55}}%
\newcommand{\done}{\rlap{$\square$}{\raisebox{2pt}{\large\hspace{1pt}\cmark}}%
\hspace{-2.5pt}}
\newcommand{\wontfix}{\rlap{$\square$}{\large\hspace{1pt}\xmark}}
% EXAMPLE CODE FOR TODO LIST IN S TODO BOX ENVIRONMENT:
%\begin{todobox}
%  \begin{todolist}
%  \item[\done] Frame the problem
%  \item Write solution
%  \item[\wontfix] profit
%  \end{todolist}
%\end{todobox}



% Ermöglicht erweiterte Referenzen. Soweit angepasst, dass schöne Makros möglich sind (siehe Kommando-Definitionen)
\usepackage[english]{varioref}
\renewcommand*{\reftextfaraway}[1]{auf S.\,\pageref{#1}}%
\renewcommand*{\reftextcurrent}{\unskip}%
% generell neu: \vpageref, \vref und \vref*, diese Referenzierungen geben die Seite mit an

% Ermöglichung von Unterabbildungen
%\usepackage{subfig}
%% Verwendung in figure: \subfloat[Optionaler Verzeichnistext][Unterschrift \label{sfig:bla}]{...Zeug...}
% I use subcaption
\usepackage{subcaption}
\usepackage{caption}
\captionsetup{width=.8\textwidth}

% Das normgerechte Eurozeichen
\usepackage{eurosym}
% Verwendung mit \euro

% Kommandos nach Ende einer Seite absetzen
\usepackage{afterpage} 
% Wichtigste Anwendung ist das Flushen von Floats: \afterpage{\clearpage}

 % Nassi-Schneidermann Unterstützung
\usepackage{Packets/nassi}
\setiftext{W}{F}
\nassiwidth=\textwidth
% Wichtige Befehle:
% \NSD{}
% \WHILE{Text}{} \ENDWHILE
% \ACTION{Text}

% Einrahmen von Text
\usepackage{framed}
\addtolength{\FrameRule}{1pt} % dickeren Rahmen
% Verwendung: \begin{framed} ... \end{framed}

%Comment blocks using \begin{comment} ... \end{comment}
\usepackage{comment}
%\includecomment{commentnotes} % environment to comment stuff
\excludecomment{commentnotes}

% For color boxes like todobox, notebox, etc
\usepackage[most]{tcolorbox}


% Abstrct
\usepackage{abstract}
\usepackage{lipsum}